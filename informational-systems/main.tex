\documentclass[12pt, a4paper]{article}

% Betöltjük a preamble.tex fájlt. Fontos a relatív elérési út!
\usepackage{booktabs}
\usepackage[T1]{fontenc}
\usepackage[utf8]{inputenc}
\usepackage[english]{babel}
\usepackage{amsmath, amssymb, amsthm} % Matek környezetek
\usepackage{geometry}
\geometry{
    a4paper,
    margin=2.5cm,
}
\usepackage{xcolor}
\usepackage{tikz}
\usepackage{float}
\usepackage{caption}
\usetikzlibrary{calc, positioning, arrows.meta, shapes.geometric, fit, matrix}

\tikzset{
    state/.style={
        matrix of nodes,
        nodes={
            draw=black!50, 
            minimum size=0.65cm, 
            anchor=center, 
            fill=white, 
            font=\ttfamily\footnotesize
        },
        column sep=-\pgflinewidth, 
        row sep=-\pgflinewidth,
        draw=black, 
        thick,
        inner sep=0pt
    },
    process/.style={rectangle, draw=black, thick, fill=white, align=center, minimum width=3.5cm, minimum height=0.8cm},
    block/.style={rectangle, draw=blue!80!black, thick, fill=blue!5, minimum width=2cm, minimum height=1cm, align=center, font=\bfseries},
    enc/.style={trapezium, trapezium angle=70, draw=black, thick, fill=gray!10, minimum width=2cm, align=center, shape border rotate=270},
    xor/.style={circle, draw=black, thick, inner sep=0pt, minimum size=0.4cm, path picture={\draw[thick] (path picture bounding box.north) -- (path picture bounding box.south) (path picture bounding box.west) -- (path picture bounding box.east);}},
    group/.style={rectangle, draw=gray, dashed, inner sep=0.3cm},
    arrow/.style={-Latex, thick},
    label/.style={font=\small\bfseries, color=gray!80!black, align=center},
    rsaStep/.style={rectangle, draw=black, thick, fill=white, align=center, minimum width=4cm, minimum height=0.8cm, rounded corners},
    data/.style={rectangle, draw=blue!80!black, thick, fill=blue!5, align=center, minimum width=2.5cm},
    arrow/.style={-Latex, thick},
    actor/.style={circle, draw=black, thick, minimum size=1.5cm, fill=gray!10, font=\bfseries},
    box/.style={rectangle, draw=black, thick, fill=white, align=center, minimum width=2.5cm, minimum height=1.5cm},
    op/.style={rectangle, draw=black, thick, fill=white, align=center, minimum width=3cm},
    decision/.style={diamond, draw=orange!80!black, thick, fill=orange!5, aspect=2, align=center, font=\small, inner sep=1pt},
    term/.style={rectangle, draw=black, thick, fill=gray!20, rounded corners, minimum height=0.8cm},
    elgamalStep/.style={rectangle, draw=black, thick, fill=white, align=center, minimum width=4cm, minimum height=0.8cm, rounded corners},
    secret/.style={rectangle, draw=red!80!black, thick, fill=red!10, align=center},
    public/.style={rectangle, draw=green!60!black, thick, fill=green!10, align=center},
    random/.style={diamond, draw=orange!80!black, thick, fill=orange!10, aspect=2, inner sep=2pt, font=\small, align=center},
    sigStep/.style={rectangle, draw=black, thick, fill=white, align=center, minimum width=4cm, minimum height=0.8cm, rounded corners},
    check/.style={diamond, draw=orange!80!black, thick, fill=orange!10, aspect=2, inner sep=2pt, font=\small, align=center},
    hash/.style={trapezium, draw=purple!80!black, thick, fill=purple!10, shape border rotate=270, align=center, minimum width=2cm}
}

\definecolor{kiemelt}{HTML}{0078A8}
\newcommand{\kiemeles}[1]{\textcolor{kiemelt}{\textbf{#1}}}

% Tételstílusok
\newtheoremstyle{tetelstilus}
    {\topsep}
    {\topsep}
    {\itshape}
    {}
    {\bfseries}
    {.}
    {.5em}
    {\thmname{#1}\thmnumber{ #2}\thmnote{ (\textbf{#3})}}
\theoremstyle{tetelstilus}

\newtheorem{defin}{Definíció}
\newtheorem{tetel}{Tétel}

\title{\LARGE\textbf{Informational Systems}}




\begin{document}
\maketitle
\thispagestyle{empty} % A főcím oldalán eltávolítja az oldalszámot



% Global TikZ Styles
\tikzset{
    vector/.style={-Latex, very thick, blue},
    result/.style={-Latex, very thick, red},
    axis/.style={-Latex, thick, black},
    grid/.style={gray!30, very thin},
    matrixbox/.style={rectangle, draw=black, thick, fill=white, align=center, minimum width=1cm, minimum height=1cm}
}


\author{Molnar Botond}
\date{}


\maketitle
\tableofcontents
\newpage


\section{Data Models and its Implementations}

\subsection{Conceptual Data Model}

The conceptual data model describes the database at a very high level and is useful to understand the needs or requirements of the database. It is this model, that is used in the requirement-gathering process i.e. before the Database Designers start making a particular database. One such popular model is the entity/relationship model (ER model). The E/R model specializes in entities, relationships, and even attributes that are used by database designers. In terms of this concept, a discussion can be made even with non-computer science(non-technical) users and stakeholders, and their requirements can be understood.

\textbf{Entity-Relationship Model( ER Model)}: It is a high-level data model which is used to define the data and the relationships between them. It is basically a conceptual design of any database which is easy to design the view of data.

Components of the ER model:

\begin{itemize}
  \item \textbf{Entity}: An entity is referred to as a real-world object. It can be a name, place, object, class, etc. These are represented by a rectangle in an ER Diagram. 
  \item \textbf{Attributes}: An attribute can be defined as the description of the entity. These are represented by Ellipse in an ER Diagram. It can be Age, Roll Number, or Marks for a Student. 
  \item \textbf{Relationship}: Relationships are used to define relations among different entities. Diamonds and Rhombus are used to show Relationships. 
\end{itemize}

\subsubsection{Characteristics}

\begin{itemize}
  \item Offers Organization-wide coverage of the business concepts. 
  \item This type of Data Models are designed and developed for a business audience. 
  \item The conceptual model is developed independently of hardware specifications like data storage capacity, location or software specifications like DBMS vendor and technology. The focus is to represent data as a user will see it in the “real world.” 
\end{itemize}

\subsection{Representational Data Model}

This type of data model is used to represent only the logical part of the database and does not represent the physical structure of the database. The representational data model allows us to focus primarily, on the design part of the database. A popular representational model is a Relational model. The relational Model consists of Relational Algebra and Relational Calculus. In the Relational Model, we basically use tables to represent our data and the relationships between them. It is a theoretical concept whose practical implementation is done in Physical Data Model. 

The advantage of using a Representational data model is to provide a foundation to form the base for the Physical model.

\subsubsection{Characteristics}

\begin{itemize}
  \item Represents the logical structure of the database.
  \item Relational models like Relational Algebra and Relational Calculus are commonly used.
  \item Uses tables to represent data and relationships.
  \item Provides a foundation for building the physical data model.
\end{itemize}

\subsection{Physical Data Model}

The physical Data Model is used to practically implement Relational Data Model. Ultimately, all data in a database is stored physically on a secondary storage device such as discs and tapes. This is stored in the form of files, records, and certain other data structures. It has all the information on the format in which the files are present and the structure of the databases, the presence of external data structures, and their relation to each other. Here, we basically save tables in memory so they can be accessed efficiently. In order to come up with a good physical model, we have to work on the relational model in a better way. Structured Query Language (SQL) is used to practically implement Relational Algebra.

This Data Model describes HOW the system will be implemented using a specific DBMS system. This model is typically created by DBA and developers. The purpose is actual implementation of the database.

\subsubsection{Characteristics}

\begin{itemize}
  \item The physical data model describes data need for a single project or application though it maybe integrated with other physical data models based on project scope. 
  \item Data Model contains relationships between tables that which addresses cardinality and nullability of the relationships. 
  \item Developed for a specific version of a DBMS, location, data storage or technology to be used in the project. 
  \item Columns should have exact datatypes, lengths assigned and default values. 
  \item Primary and Foreign keys, views, indexes, access profiles, and authorizations, etc. are defined 
\end{itemize}


\section{Relational Model}

\begin{itemize}
    \item \textbf{Tulajdonságtípus}:
    \begin{itemize}
        \item Azonos szerepű tulajdonságok \textbf{absztrakciója}
    \end{itemize}
    \item \textbf{Egyedtípus}:
    \begin{itemize}
        \item Azonos tulajdonságtípusokkal rendelkező egyedek \textbf{absztrakciója}
    \end{itemize}
    \item \textbf{Kapcsolattípus}:
    \begin{itemize}
        \item Két vagy több egyedtípus között fennálló, jól meghatározott viszony
    \end{itemize}
\end{itemize}

\textbf{Tulajdonságtípus lehet:}

\begin{itemize}
    \item \textbf{Összetettség szerint}:
    \begin{itemize}
        \item atomi
        \item összetett
    \end{itemize}
    \item \textbf{egyszerre felvett értékek szerint}:
    \begin{itemize}
        \item egyértékű
        \item halmazértékű
    \end{itemize}
    \item \textbf{megjelenés szerint}:
    \begin{itemize}
        \item tárolt
        \item származtatott
    \end{itemize}
\end{itemize}

\textbf{Kapcsolat:}

\begin{itemize}
    \item \textbf{Szorossága}:
    \begin{itemize}
        \item kötelező
        \item félig kötelező (egy oldalnak)
        \item opcionális
    \end{itemize}
    \item \textbf{Fokszám}:
    \begin{itemize}
        \item hány egyedtípus vesz részt
    \end{itemize}
\end{itemize}

Conceptional, logical and physical modelling levels separate.


$D$ domain is a set of atomic values (including NULL)

\begin{itemize}
  \item name
  \item type
  \item format
  \item constraints on values
\end{itemize}


The relational schema $R$: $R(A_1, A_2, ..., A_n)$

\begin{itemize}
  \item $R$ is the name of the schema
  \item $A_1, A_2, ..., A_n$ are the attributes
\end{itemize}

The $A_i$ attribute can have values from the $D_i$ domain (the domain contains the possible values of the attribute)


The degree of a relation is the number of attributes in the realtion.

\begin{figure}[H]  % or [!htbp] if you’re fine with floating
  \centering
  \includegraphics[width=1\textwidth]{relational_model.png} % adjust width/height as needed
  \caption{Relational model overview}
  \label{fig:relational-model}
\end{figure}

\subsection{Definition of the Relation}

$R$'s defining domains Descaret-product's subset \newline
Az R-et meghatározó tartományok Descartes szorzatának a részhalmaza

$$ r(R) \subset (dom(A_1) \times dom(A_2) \times ... \times dom(A_n)) $$

\subsubsection{The Other Definition} 

\begin{itemize}
    \item \textbf{A reláció}
    \begin{itemize}
        \item Az $R(A_1, A_2, \dots, A_n)$ relációséma egy $r$ relációja -- amit szokás $r(R)$-rel is jelölni -- elem $n$-eseknek egy halmaza: $r = \{t_1, t_2, \dots, t_m\}$
        \item Minden $t_i$ elem ($1 \le i \le m$) $n$ darab értéknek egy rendezett listája: $t_i = \langle v_{i1}, v_{i2}, \dots, v_{in} \rangle$, ahol
        \item minden $v_{ij}$ érték ($1 \le j \le n$) vagy $A_j$ tartományának az eleme.
    \end{itemize}
\end{itemize}

The order of the tuples in a relation cannot be interpreted since they are part of a set.

\vspace{1cm}

A record's values must be atomic, so they cannot be complex or set values, therefore every relation is in 1NF.

\subsection{The NULL value}

\begin{itemize}
  \item We do not know if the value exists
  \item The value exists but we do not know it
  \item The value does not exists for the given incidence 
\end{itemize}


\section{Integrity Constraints}

\subsection{Domain Constraint}

The domain constrain states that for every record for every value belonging to
the $A$ attribute, it must originate from $dom(A)$, and these values must be atomic

Data types:

\begin{itemize}
  \item numerical
  \item integer
  \item real
  \item character
  \item logic
  \item string
  \item date
  \item other special data types (time, currency, timestamp)
\end{itemize}


\subsection{Key Constrains}

By definition in a relation every record is different, therefore in a relation there is no 2 records where all the attribute values are the same.


\subsubsection{Superkey}

Az $R$ relációsémának létezik egy olyan $SK$ attribútumhalmaza, amely
olyan tulajdonságú, hogy tekintve $R$ bármelyik $r$ relációját, az adott
relációban nincs két olyan rekord, amelynek az értékei azonosak
lennének ezen $SK$ attribútumokra vonatkozóan. Azaz bármely két
különböző $t1$ és $t2$ rekordot kiválasztva $R$ egy $r$ relációjából:
$t1[SK] \ne t2[SK]$.
Minden ilyen $SK$ attribútumhalmaz az $R$ relációséma szuperkulcsa.


Every relation has a superkey: the set of all the attributes.

\subsubsection{Key}

Egy $R$ relációséma $K$ kulcsa $R$-nek egy olyan szuperkulcsa, amelyből egy $A$
attribútumot elhagyva, az így kapott $K'$ attribútumhalmaz már nem
szuperkulcsa $R$-nek.


Satisfies the following two condition:

\begin{itemize}
  \item Bármilyen relációt tekintve, a reláció két különböző rekordjának nem
lehetnek azonosak a kulcsban szereplő attribútumokhoz tartozó
értékei.
  \item Minimális szuperkulcs, azaz egy olyan szuperkulcs, amelyből nem
tudunk úgy eltávolítani egyetlen attribútumot sem, hogy az
egyediségre vonatkozó feltétel továbbra is fennálljon.
\end{itemize}

$K$ key is simple if it only consists of one attribute, else it is complex.


A relational scheme can have multiple keys, these are candidate keys.

The modeller's job to choose a \textbf{primary key} from the candidate keys, its values will identify each row in the relation

We can put a \textit{unique} constraint to the other candidate keys.


\subsection{Entity Integrity Constarint (Egyedintegritasi megszoritas)}

It states that no single primary key value can be NULL value. If the primary key is complex, none of its components can be NULL value.


\subsection{Referential Integrity Constraint (Hivatkozasintegritasi megszoritas)}

A hivatkozási integritási megszorítást két reláció között értelmezzük, és a két relációban lévő rekordok között konzisztencia megteremtése érdekében használjuk.

Egy $R_1$ relációséma $FK$-val jelölt attribútumhalmaza \textbf{külső} (idegen) \textbf{kulcsa} $R_1$-nek, amely hivatkozik az $R_2$ relációsémára, ha eleget tesz a következő feltételeknek:

\begin{itemize}
  \item Az $FK$-beli attribútumoknak és az $R_2$ $PK$-val jelölt elsődleges kulcsattribútumainak páronként azonos a tartománya; ekkor azt mondjuk, hogy az $FK$ attribútumok hivatkoznak az $R_2$ relációsémára.
  \item Bármely $r_1(R_1)$ aktuális állapotának egy $t_1$ rekordjában egy $FK$-beli érték vagy megjelenik egy $r_2(R_2)$ aktuális állapotának valamely $t_2$ rekordjában $PK$ értékeként, vagy az értéke NULL. Az előbbi esetben $t_1[FK] = t_2[PK]$, ekkor azt mondjuk, hogy a $t_1$ rekord hivatkozik a $t_2$ rekordra.
\end{itemize}

Ha e két feltétel teljesül, egy hivatkozási integritási megszorítás áll fenn $R_1$-ről $R_2$-re vonatkozóan.

Egy relációs adatbázisséma minden integritási megszorítását meg kell határozni.


\subsection{Other Type of Constraints}

\begin{itemize}
  \item NULL constraint: Can an attribute's values be NULL
  \item Semantic integrity constraints: Enforced by software, trigger
  \item Dependency between data: \textit{functional dependency, multivalue dependency}
  \item State constraints
  \item Dynamic constraints: E. g. a workers' pay can only grow
\end{itemize}

\section{Relational Database and Relational Database Schema}

\begin{itemize}
  \item \textbf{Relational Database Scheme}: $S$ relational database scheme is the $S = \{R_1, R_2, ..., R_m\}$ relational scheme set and the integrity contraint set.
  \item \textbf{Relational Database (State)}: $S$'s one relational database (state) is such $DB = \{r_1, r_2, ..., r_m\}$ relation sates set, that every $r_i$ is a relation of $R_i$ and every $r_i$ satisfies the integrity constraints given in $IC$
\end{itemize}


\subsection{Operations on a Relation}

\begin{itemize}
  \item Query
  \item Insert
  \item Modify
  \item Delete
\end{itemize}

\begin{lstlisting}[style=sqlcode,caption={Combining queries on employees and departments},label={lst:employees-union}]
SELECT * FROM employees e, departments d
WHERE e.email = 'SSTILES'
  AND e.department_id = d.department_id
UNION ALL
SELECT * FROM employees e, departments d
WHERE d.department_name = 'Treasury'
  AND e.department_id = d.department_id;
\end{lstlisting}


\subsection{ACID Properties}

\begin{itemize}
  \item \textbf{Atomicity}: All or nothing execution.
  \item \textbf{Consistency}: Data remains valid after transactions.
  \item \textbf{Isolation}: Concurrent transactions do not interfere with each other.
  \item \textbf{Durability}: Committed data is saved permanently.
\end{itemize}



\section{Object Relational Model}

\subsection{Weaknesses of RDBMS}

\begin{itemize}
  \item A való világbeli egyedeket szegényesen ábrázolja: A relációk nem tükrözik a való világot
  \item Szemantikus túlterhelés
  \begin{itemize}
    \item Csak reláció van
    \item a kapcsolatoknak nem lehet jelentést adni
    \item a kapcsolatokat nehéz megkülönböztetni az egyedektől
  \end{itemize}
  \item Az integritás és a vállalatszintű megszorítások szegényes
támogatása
  \begin{itemize}
    \item Az integritást megszorításokban fejezi ki
    \item Sok rendszer nem biztosítja ezek kikényszerítését, ami miatt be
kell építeni őket az alkalmazásba
  \end{itemize}
  \item Homogén adat struktúra
  \begin{itemize}
    \item Minden sornak ugyanazok az attribútumai
    \item Az oszlop minden értékének ugyanabból a tartományból
kell származnia
    \item A mezők értékei csak atomiak lehetnek
  \end{itemize}
  \item Korlátozott műveletek
  \begin{itemize}
    \item Ami az SQL specifikációban van
    \item Pl: GIS-ben pontok, vonalak, poligonok vannak tárolva és a
művelet szükséges a távolság, a metszet vagy a
tartalmazás lekérdezésére
  \end{itemize}
  \item A rekurzív lekérdezéseket nehezen kezeli
  \item Nehézkesek a sémaváltoztatások
\end{itemize}


\subsection{ORDBMS}

\begin{itemize}
  \item A relációs adatbázisok beépítették az OO világot
  \begin{itemize}
    \item Típus rendszer
    \item Egységbezárás
    \item Öröklés
    \item Polimorfizmus
    \item A metódusok dinamikus kötése
    \item Összetett objektumok (nem 1NF-ben lévő objektumok)
    \item Objektum azonosító.
  \end{itemize}
\end{itemize}


Advantages:

\begin{itemize}
  \item Az újrafelhasználás és a megosztás; Pl: A GIS-ben a
távolságot egyszer definiáltuk, akkor bárki újrahívhatja
  \item Megtartja a relációs tudást
\end{itemize}

Disadvantages:

\begin{itemize}
  \item Összetett és drága
\end{itemize}


\subsection{User Defined Type (UDT)}

Normális eseten egy tábla oszlopának az adattípusa, egy
SQL-ben meghívott rutin egy SQL változójának a típusa
vagy egy külsőleg meghívott SQL rutin egy paramétere.

A felhasználó által definiált típus egy olyan típus, amely
nincs beépítve az adatbázisrendszerbe vagy a
programozási nyelvbe, de egy alkalmazásfejlesztés
részeként lehet definiálni, amely gyakran a
viselkedésének a leírásával is jár.

\subsubsection{Distinct Type}

Egy egyszerű beépített adattípuson alapszik, mint pl.
integer, de ezeknek az értékeit nem lehet közvetlenül
összekeverni az eredeti alaptípussal (cast szükséges).

Első osztályú típusok, azaz lehet őket használni
oszlopdefinícióban, változó deklarációban, stb, mint
bármely más SQL beépített típust.

\begin{lstlisting}[style=sqlcode,label={lst:shoe-size-type}]
CREATE TYPE shoe_size AS INTEGER FINAL;
\end{lstlisting}

A \textit{Final} kulcsszót meg kell adni a distinct type definíciókban.
Adjuk meg a FINAL kulcsszót, ha a típusnak nem lehet
altípusa.




\subsection{Structured UDT}

Belső struktúrája van

\begin{lstlisting}[style=sqlcode,label={lst:shoe-size-type}]
Create type address
(Street_name varchar2(50),
Apartment_number varchar2(5),
City varchar2(50),
Country varchar2(50),
Postal_code varchar2(10));
\end{lstlisting}

Két fő jellemző:

\begin{itemize}
  \item Adatok, amelyeket alapvetően tárolnak
  \item És műveletek, amelyeket az adatokon lehet
végrehajtani. A kódjuk implementációját a típus
definiálója adja meg.
\end{itemize}



Attribútumok

\begin{itemize}
  \item Minden attribútumnak van egy egyszerű típusa (amely nincs
korlátozva az SQL beépített atomi típusaira)
  \item Egy típus attribútumainak kollekcióját a típus
reprezentációjának hívják (representation rész)
  \item A Java objektumok mezőinek felel meg az SQL strukturált
típusok attribútumai.
\end{itemize}


\subsubsection{Viselkedés és szemantika}

Rutinok segítségével lehet megadni őket (metódusok,
eljárások, függvények)


Az SQL lehetővé teszi, hogy a típusok tervezői viselkedést
adjanak meg valamilyen nyelven megírt rutinok
segítségével

\subsubsection{Öröklődés}

Az altípus egy olyan adattípust ír le, amely birtokol minden
olyan jellemzőt, amellyel egy másik típus

A szupertípus olyan típust ír le, amelynek a jellemzőit egy
altípus birtokolja.


\subsubsection{Observer és Mutator metódusok}

Egy strukturált típus minden attribútumának van két
beépített, a rendszer által definiált metódusa:

\begin{itemize}
  \item \textbf{Observer} metódus: Az attribútum értékével tér vissza.
  \item \textbf{Mutator} metódus: lehetővé teszi, hogy az attribútum értéke
változzon
\end{itemize}


\subsubsection{Konstruktorok}

A típus egy példányát hozza létre, más néven inicializáló metódus.

\section{XML Databases}

Extensible Markup Language (XML) is a standard for storing and transporting data. Unlike relational databases which store data in rigid rows and columns, XML databases are designed to handle \textbf{semi-structured data}.

\subsection{Key Characteristics}
\begin{itemize}
    \item \textbf{Hierarchical Structure:} Data is stored in a tree-like structure (DOM - Document Object Model) consisting of a root element, child elements, and attributes.
    \item \textbf{Self-Describing:} The schema (tags) travels with the data. A document can be understood without referencing an external catalog (though XSD schemas are used for validation).
    \item \textbf{Flexibility:} New fields (tags) can be added without restructuring the entire database.
\end{itemize}

\subsection{Data-Centric vs. Document-Centric}

Understanding the nature of the data is the first step in XML database design.

\begin{enumerate}
    \item \textbf{Data-Centric XML:}
    \begin{itemize}
        \item Used for data transport (e.g., SOAP messages, REST API responses).
        \item Highly structured with predictable fields.
        \item The order of sibling elements often does not matter.
        \item \textit{Example:} Flight schedules, Stock quotes.
    \end{itemize}
    
    \item \textbf{Document-Centric XML:}
    \begin{itemize}
        \item Used for mixed content (text with markup).
        \item The order of elements (structure) is critical to the meaning.
        \item \textit{Example:} A book, a legal contract, or a medical record (HL7).
    \end{itemize}
\end{enumerate}

\subsection{Database Architectures}

There are two primary ways to implement an XML database system.

\subsubsection{1. XML-Enabled Databases (XEDB)}
These are traditional Relational Databases (like Oracle, SQL Server, PostgreSQL) that have been extended to handle XML.

\begin{itemize}
    \item \textbf{Storage:} They often store XML in a generic \texttt{CLOB} (Character Large Object) column or a specialized binary XML type.
    \item \textbf{Mapping:} Middleware maps the XML nodes to relational tables ("Shredding").
    \item \textbf{Pros:} Leverages existing ACID properties, security, and transaction management of established RDBMS.
    \item \textbf{Cons:} Performance overhead when parsing deep XML trees; impedance mismatch between the tree model and table model.
\end{itemize}

\subsubsection{2. Native XML Databases (NXD)}
Databases specifically designed to store XML documents as the fundamental unit of logical storage (e.g., BaseX, eXist-db, MarkLogic).

\begin{itemize}
    \item \textbf{Storage:} Data is stored physically in a format that closely resembles the logical tree structure (e.g., using persistent DOM or proprietary binary tree formats).
    \item \textbf{No Mapping:} There is no conversion to tables. The database model is the XML document itself.
    \item \textbf{Pros:} Faster retrieval for complex hierarchies; precise document order preservation.
    \item \textbf{Cons:} Lack of standardized tooling compared to SQL; smaller market share.
\end{itemize}

\subsection{Query Languages}

Just as SQL is the standard for RDBMS, XML databases utilize specific W3C standards for data retrieval.

\subsubsection{XPath (XML Path Language)}
XPath is used to navigate through elements and attributes in an XML document. It uses a path-like syntax similar to file systems.

\textbf{Example Data (books.xml):}
\begin{lstlisting}[language=XML]
<bookstore>
  <book category="cooking">
    <title lang="en">Everyday Italian</title>
    <author>Giada De Laurentiis</author>
    <year>2005</year>
    <price>30.00</price>
  </book>
</bookstore>
\end{lstlisting}

\textbf{XPath Expressions:}
\begin{itemize}
    \item \texttt{/bookstore/book[1]/title}: Selects the title of the first book.
    \item \texttt{//author}: Selects all author nodes anywhere in the document.
    \item \texttt{/bookstore/book[price>35]}: Selects books where the price is greater than 35.
\end{itemize}

\subsubsection{XQuery}
XQuery is the functional query language for XML. It is a superset of XPath and allows for complex logic, joining, and reshaping of data (similar to how SQL allows \texttt{SELECT ... WHERE}). It relies on the \textbf{FLWOR} expression:

\begin{itemize}
    \item \textbf{F}or: Iterates over a sequence of nodes.
    \item \textbf{L}et: Binds variables.
    \item \textbf{W}here: Filters results.
    \item \textbf{O}rder by: Sorts results.
    \item \textbf{R}eturn: Constructs the result structure.
\end{itemize}

\textbf{XQuery Example:}
\begin{lstlisting}[language=SQL]
(: Select titles of books costing more than $29 :)
for $x in doc("books.xml")/bookstore/book
where $x/price > 29
order by $x/title
return <result>{ $x/title }</result>
\end{lstlisting}

\subsection{Summary comparison}

\begin{table}[ht]
\centering
\begin{tabular}{@{}lp{5cm}p{5cm}@{}}
\toprule
\textbf{Feature} & \textbf{Relational DB} & \textbf{Native XML DB} \\ \midrule
\textbf{Data Model} & Tables, Rows, Columns & Trees, Nodes, Elements \\
\textbf{Structure} & Rigid Schema & Semi-structured / Schema-less \\
\textbf{Order} & Unordered sets & Intrinsic order matters \\
\textbf{Query Language} & SQL & XQuery / XPath \\
\textbf{Integrity} & Referential Integrity (PK/FK) & ID/IDREF (weaker) \\ \bottomrule
\end{tabular}
\caption{Comparison of Relational and Native XML Databases}
\end{table}


\section{NoSQL Databases}

Két fő ok miatt születtek meg:

\begin{itemize}
  \item A nagy adattömeg kezelése kikényszerítette
  azt, hogy klaszterezéssel kapcsoljanak
  össze gépeket, és így nagy hardver
  platformokat építsenek.
  \item Ez az igény azt is megnehezítette, hogy az
  alkalmazáskódok jól működjenek a relációs
  modellel.
\end{itemize}


A két ok miatt használunk NoSQL adatbázist:

\begin{itemize}
  \item Az alkalmazásfejlesztés termelékenysége miatt: sok idő és
  erőfeszítés megy el arra, hogy a memóriastruktúrákat leképezzék
  relációs adatbázisra. A NoSQL adatbázisok olyan adatmodellt
  biztosítanak, amelyek jobban alkalmazkodnak az alkalmazások
  szükségleteihez. Leegyszerűsödik az adatbázis elérése, kevesebb
  kódot kell írni, nyomkövetni és javítani.
  \item A nagymennyiségű adat miatt: ma megéri több adatot tárolni és
sokkal gyorsabban feldolgozni. Relációs adatbázissal ez drága vagy
lehetetlen. A relációs adatbázisokat úgy tervezték, hogy egy gépen
fussanak, azonban ma már általában gazdaságosabb sok kisebb,
olcsóbb gép klaszterén végezni a nagymennyiségű adat
feldolgozását. Sok NoSQL adatbázist úgy terveztek, hogy
klasztereken futnak.
\end{itemize} 


A fő kategóriák:

\begin{itemize}
  \item Kulcs-érték (key-value) adatbázisok
  \item Dokumentum adatbázisok
  \item Oszlopcsalád (column-family) adatbázisok
  \item Gráf adatbázisok
  \item Objektum adatbázisok
  \item XML adatbázisok
\end{itemize}



Vannak nem tiszta adatbázisok is, amelyek kevernek
két kategóriát.

\subsection{Aggregátum adatmodellek}

A kulcs-érték, a dokumentum és az
oszlopcsalád adatmodelleket együtt
aggregátum adatmodelleknek hívjuk, a
közös jellemzőik miatt.

Az aggregátumban kulcs-érték párok
vannak, ahol az érték lehet
dokumentum, egyszerű adat, vagy
valamilyen összetettebb struktúra, mint
halmaz, lista, stb.

Azt támogatja, hogy gyakran akarunk
olyan adatokon dolgozni, amelyeket
olyan unitokba akarunk szervezni,
amelynek bonyolultabb a struktúrája,
mint amelyet a listák és a
rekordstruktúrák egymásba ágyazása
lehetővé tenne.

Az \textbf{aggregátum} kapcsolódó objektumok
egy olyan gyűjteménye, amelyeket egy
egységként szeretnénk kezelni. Ez
egyben az adatmódosítás és a
konzisztencia menedzsment egysége.
Általában atomi műveletekkel szeretjük
módosítani az aggregátumokat és
aggregátumban kifejezve szeretünk
kommunikálni az adattárolóval.

Az aggregátumok megkönnyítik az
adatbázisoknak a klaszteren való
műveletek kezelését, mert az
aggregátum a replikáció és a sharding
természetes egysége.

Az aggregátumok megkönnyítik a
programozók dolgát is, mert az gyakran
az aggregátumokon keresztül
módosítják.


\begin{figure}[H]
  \centering
  \includegraphics[width=0.8\textwidth]{aggregate_example.png}
  \caption{Example of an aggregate structure in a NoSQL context.}
  \label{fig:aggregate-example}
\end{figure}

\begin{lstlisting}[style=sqlcode,language=JSON,caption={Aggregate stored as nested JSON},label={lst:aggregate-json}]
{
  "customer": {
    "id": 1,
    "name": "Martin",
    "billingAddress": [
      {
        "city": "Chicago"
      }
    ],
    "orders": [
      {
        "id": 99,
        "customerId": 1,
        "orderItems": [
          {
            "productId": 27,
            "price": 32.45,
            "productName": "NoSQL Distilled"
          }
        ],
        "shippingAddress": [
          {
            "city": "Chicago"
          }
        ],
        "orderPayment": [
          {
            "ccinfo": "1000-1000-1000-1000",
            "txnId": "abelif879rft",
            "billingAddress": {
              "city": "Chicago"
            }
          }
        ]
      }
    ]
  }
}
\end{lstlisting}


Nincs univerzális válasz arra a kérdésre, hogy hol
húzzuk meg az aggregátumok határát. Csak attól
függ, hogy hogyan fogjuk módosítani az adatot.

Ha a customer-t az orders-sel együtt fogjuk
módosítani, akkor egy aggregátummal fogunk
dolgozni.

Ha módosításkor egy megrendelésre fókuszálunk,
akkor érdemes szétválasztani az
aggregátumokat.

Gyakran előfordul, hogy egy alkalmazás mindkét
megközelítést is használná.

\subsubsection{Előnyei}

Nagyszerűen segít a klaszteren
való futtatást, ahol
minimalizálnunk kell az adatok
lekérdezéshez szükséges
csomópontok számát. Az
aggregátumokkal egyértelműen
megmondjuk az adatbázisnak,
hogy mely adatokat fogjuk
együtt módosítani. Ezeknek az
adatoknak egy csomóponton
kell lenniük.


\subsubsection{Hátrányai}

\begin{itemize}
  \item Gyakran nehéz meghúzni a
határt az aggregátumok között,
különösen akkor, amikor
ugyanazt az adatot több
különböző esetben használjuk.
  \item Az aggregátum struktúra sok
típusú adatlekérdezésben 
módosításban segíthet,
azonban más típusú
adatlekérdezésben 
módosításban gátolhat.
\end{itemize}


Gyakran azt mondják, hogy a NoSQL
adatbázisok nem támogatják az ACID
tranzakciókat, és így feláldozza a konzisztenciát.

Általában igaz, hogy az aggregátumorientált
adatbázisoknak nincs olyan ACID tranzakciójuk,
amely több aggregátumra kiterjed.

Helyette \textbf{egy aggregátum egyetlen atomi
módosítását} támogatják. Ez azt jelenti, hogy ha
több aggregátumot szeretnénk atomi módon
módosítani, azt nekünk az alkalmazáskódunkban
kell kezelni.

\subsection{Kulcs- érték és dokumentum adatmodellek}

Mindkét típusú adatbázis sok aggregátumból épül fel,
ahol minden aggregátumnak van egy kulcsa vagy egy
azonosítója, amelyet az adat elérésére használunk.

\begin{table}[H]
\centering
\renewcommand{\arraystretch}{1.5} % Adjusts row height for readability
\begin{tabular}{p{0.45\textwidth} p{0.45\textwidth}}
\toprule
\textbf{Kulcs-érték adatbázisok} & \textbf{Dokumentum adatbázisok} \\
\midrule
% Row 1
$\bullet$ Az aggregátum \textbf{átlátszatlan} az adatbázisban. Az átlátszatlanság előnye az, hogy azt tárolhatunk az aggregátumban amit csak szeretnénk. Az adatbázisnak lehet, hogy van valamilyen méretkorlátozása, de ez több, mint ami a korlátozná a szabadságunkat. & 
$\bullet$ Képes látni az aggregátum \textbf{struktúráját}. A dokumentum adatbázis korlátozásokat adhat arra nézve, hogy milyen struktúrákat és típusokat helyezhetünk el egy aggregátumba. Cserébe az elérésnél nagyobb rugalmasságot biztosít. \\

% Row 2
$\bullet$ Egy aggregátumot csak a kulcsán keresztül kereshetünk ki. & 
$\bullet$ Lekérdezéseket írhatunk az aggregátum mezői alapján, kinyerhetjük az aggregátumok részeit a teljes aggregátum helyett, és indexeket hozhatunk létre az aggregátum tartalma alapján. \\
\bottomrule
\end{tabular}
\end{table}


\subsection{Oszlopcsalád tárak}

A legtöbb (relációs) adatbázisnak a tárolási alapegysége a
sor. Ez az írási teljesítményt támogatja. Azonban sok olyan
forgatókönyv van, ahol az írás kevés, de gyakori az olyan
lekérdezés, ahol sok sornak néhány oszlopát kérjük le. Az
ilyen esetekben a legjobb megoldás az, ha a sorokhoz
oszlopok csoportjait tároljuk, és ezek a csoportok lesznek
a tárolás egységei. Emiatt hívják őket oszloptáraknak.


A legkönnyebb talán úgy megérteni az oszlopcsalád modellt,
mint egy kétszintű aggregátumstruktúrát. Úgy, mint a
kulcs-érték táraknál, az első kulcsot gyakran a sor
azonosítójaként értelmezik, amellyel a keresett
aggregátumot meg lehet fogni. A különbség az, hogy az
oszlopcsalád táraknál a soraggregátum önmagában
részletesebb értékek egy leképezését formázza. Ezeket a
második szintű értékeket hívjuk oszlopoknak.


\begin{figure}[H]
  \centering
  \includegraphics[width=0.8\textwidth]{column_family.png}
  \caption{Column-family layout illustrating rows grouped into column families.}
  \label{fig:column-family}
\end{figure}

\begin{figure}[H]
  \centering
  \includegraphics[width=0.8\textwidth]{column_family2.png}
  \caption{Wide-column data model showing multiple column families per entity.}
  \label{fig:column-family-2}
\end{figure}

Az oszlopcsalád adatbázisok az oszlopaikat
oszlopcsaládokba szervezik. Minden oszlopnak egy
oszlopcsalád részének kell lennie. Az oszlop az elérés
egysége, azonban feltételezzük, hogy egy
oszlopcsaládot általában együtt érjük el.


Az adatokat kétféleképp strukturálhatjuk:

\begin{itemize}
  \item \textbf{Sororientáltan}: minden sor egy aggregátum (pl. customer,
amelynek az ID-ja 1234), amelynek az oszlopcsaládjai
hasznos adatcsonkokat tartalmaz az aggregátumon belül
(profile, order history).
  \item \textbf{Oszloporientáltan}: minden oszlopcsalád egy rekordtípust
definiál (pl. customer profiles), ahol minden rekordhoz több
sor tartozik. Így minden oszlopcsaládban úgy gondolhatsz
egy sorra, mint rekordok összekapcsolása.
\end{itemize}


Bármilyen oszlophoz bármilyen sort hozzá lehet
adni, így a soroknak különböző oszlopkulcsai
lehetnek. Új oszlopokat adhatunk a sorokhoz a
szokásos adatbázis használat alatt. Azonban az új
oszlopcsaládok definiálása ritka, és ehhez gyakran le
kell állítani az adatbázist.


A \textbf{sovány soroknak (Skinny rows)} kevés oszlopuk van, és sok
különböző sorban ugyanazokat az oszlopokat használják. Ebben
az esetben az oszlopcsalád egy rekordtípust definiál, minden sor
egy rekord, és minden oszlop egy mező.


A \textbf{széles sornak (wide row)} sok oszlopa van (akár több ezer), és
a soroknak nagyon különböző oszlopai vannak. Ebben az
esetben a széles oszlopcsalád egy listát modellez, ahol minden
oszlop egy elem a listában.


\subsection{Gráf adatbázisok}

A gráf adatbázisokat az olyan kis rekordok motiválták,
amelyek között bonyolult kapcsolatok állnak fenn.

Ebben az értelemben a gráf egy gráfadatstruktúra,
élekkel (edge) összekapcsolt csomópontok (node).

Ebben a struktúrában olyan kérdéseket tehetünk fel,
mint „keressük azt a könyvet az Adatbázis
kategóriában, amelyet olyan valaki írt, akit valamelyik
barátom kedvel”.

Ideális megoldás olyan esetben, amikor összetett
kapcsolatokat tartalmazó adatokat szeretnénk tárolni,
mint a szociális hálók (social networks), vagy
termékajánlások.

\begin{figure}[H]
  \centering
  \includegraphics[width=0.8\textwidth]{graph_db.png}
\end{figure}


A relációs adatbázisok a kapcsolatot külső kulcs
segítségével valósítják meg. Az összekapcsolások,
amelyek a navigációhoz szükségesek meglehetősen
drágák lehetnek, amely azt jelenti, hogy a teljesítmény
gyakran gyenge a sok kapcsolattal rendelkező
adatmodellekben.


A gráfadatbázisok navigálása a kapcsolatokon
keresztül nagyon olcsó. A gráfadatbázisok a navigálás
munkájának a nagyrészét beszúráskor végzi el és
nem a lekérdezéskor. Ez természetesen akkor
kifizetődő, amikor a lekérdezés teljesítménye sokkal
fontosabb, mint a beszúrásé.


Ezek az adatbázisok főleg egy gépen futnak és nem
klasztereken elosztva.


\subsection{Sémamentes adatbázisok}

Egyszerűen azt tárolhatjuk, amire szükségünk van.
Lehetővé teszik, hogy egyszerűen módosítsuk az
adattárunkat a projektünk előrehaladtával. Ha új dolgot
fedezünk fel, egyszerűen hozzáadhatjuk.
Ha rájövünk, hogy valamit nem szükséges tárolnunk, akkor
egyszerűen törölhetjük.

Egyszerűvé teszik, hogy nemegységes adatokkal (olyan
adatokkal, amelyeknél a rekordok más mezőkből épülnek fel)
dolgozzuk. Lehetővé teszik, hogy a rekordok pontosan azt
tartalmazzák, amire szükségünk van, nem többet és nem
kevesebbet.

A probléma: tény, hogy amikor adatokat használó programot
írunk, akkor a program majdnem mindig bízik az implicit séma
valamilyen formájában.

Attól, hogy az adatbázisunk sémamentes, általában van egy
implicit séma. Ez az implicit séma olyan feltételezések
halmaza, amelyek az adatokat manipuláló kódban lévő
adatstruktúráról szólnak.

Az alkalmazáskódban szereplő implicit séma eredményez
néhány problémát. Ha meg akarjuk érteni, hogy mit jelent az
adat, akkor az alkalmazáskódunkban kell ásni.


\subsection{Elosztási modellek}

A NoSQL adatbázisok elsődleges
érdekeltsége az a képesség volt, hogy az
adatbázis egy nagy klaszteren fusson.

Az aggregátum megközelítés jól illeszkedik a
skálázáshoz, mert az aggregátum az
elosztáshoz használt természetes egység.


Két út van az adatok elosztásához:


\begin{itemize}
  \item \textbf{Replikáció}: A replikáció ugyanazt az adatot több
csomópontra helyezi.
  \item \textbf{Sharding}: A sharding különböző adatokat helyez a
csomópontokra.
\end{itemize}



A replikáció és a sharding ortogonális
technikák: lehet használni az egyiket vagy
akár mindkettőt.

A replikáció két formában jelenhet meg:
\textbf{master-slave} és \textbf{peer-to-peer}.


Egyedülálló szerver:

\begin{itemize}
  \item A legegyszerűbb elosztási modell
  \item nincs elosztás
  \item szeretjük, mert kiküszöböli az összes komplexitást,
amellyel a többi modell megvalósítás jár
  \item A gráf adatbázisok tartoznak ebbe a kategóriába,
mivel legjobban az egy szerveres konfigurációban
működnek.
  \item Ha az adathasználatunk leginkább aggregátumok
feldolgozásáról szól, akkor egy egyszerveres
dokumentum vagy kulcs-érték tár jó választás
lehet, mert egyszerűbbek a fejleszIesztőknek.
\end{itemize}

\subsubsection{Sharding}

A sharding különböző adatokat különböző
csomópontokra tesznek, minden csomópont saját
maga végzi az írást és az olvasást.

Az aggregátumokat úgy tervezzük, hogy olyan
adatokat tartalmazzanak, amelyeket általában
együtt érünk el, így az aggregátum lesz az elosztás
nyilvánvaló egysége.


Néhány tényező javíthatja a teljesítményt:

\begin{itemize}
  \item Az adatokat ahhoz közel helyezzük el, ahol használni fogják.
  \item Próbáljuk meg úgy elhelyezni az aggregátumokat, hogy egyenlően
legyenek elosztva a csomópontok, amelyek mindegyike egyenlő
terhelést kap.
  \item Néhány esetben hasznos, ha az aggregátumokat együtt helyezzük
el, ha úgy gondoljuk, hogy sorban fogjuk felolvasni őket (A Bigtable
azt javasolja, hogy a sorokat abc sorrendben tároljuk, és a
webcímeket a megfordított domain neveik alapján rendezzük (pl:
com.martinfowler). Így több oldal adatát érhetjük el egyszerre,
amivel javítjuk a feldolgozás hatékonyságát.)
  \item Sok NoSQL adatbázis ajánl \textbf{auto-sharding}-ot, ahol az adatbázis
vállalja a felelősségét annak, hogy kiutalja a shard-ot az adathoz
és biztosítsa, hogy az adat a megfelelő shard-ra kerül.
\end{itemize}


A sharding különösen értékes a teljesítmény tekintetében,
mert az írási és az olvasási teljesítményt is növelheti. A
sharding horizontálisan skálázható írást biztosít.


\subsubsection{Master-Slave replikáció}

\begin{itemize}
  \item Több csomópontra másoljuk az adatot.
  \item Egy csomópontot kijelölünk master vagy elsődleges
csomópontnak. A master a hiteles forrása az adatoknak és
általában ő a felelős az adatok módosításáért. A többi csomópont
slave vagy másodlagos csomópont. A replikációs folyamat
szinkronizálja a slave-eket a master-rel.
  \item Master-slave replikáció leginkább akkor hasznos, amikor olvasás
intenzív adathalmazt skálázunk. Több olvasási kérés
kiszolgálásához horizontálisan skálázhatunk, ha több slave
csomópontot adunk a rendszerhez és biztosítjuk, hogy minden
olvasási kérés a slave-ekhez fusson be.
  \item Nem jó választás, ha az adathalmazunkon sok írási forgalom
történik. Korlátozva vagyunk a master írási képességével és azzal
a képességgel, hogy továbbadja a módosításokat.
  \item A másik előnye az olvasási rugalmasság: ha a master
meghibásodik, a slave-k még mindig tudják kezelni az olvasási
kéréseket. Ez akkor hasznos, ha a legtöbb adatelérés olvasás.
A master meghibásodása lehetetlenné teszi az írások
kezelését addig, amíg a master helyre nem áll vagy egy másik
mastert ki nem jelölünk. Mivel a slave-k a master másolatai, a
master meghibásodása utáni helyreállás felgyorsul, mivel egy
slave-t gyorsan ki lehet jelölni új masternek.
  \item A mastert ki lehet jelölni kézzel és automatikusan.
  \begin{itemize}
    \item A kézzel való kijelölés tipikusan azt jelenti, hogy amikor konfiguráljuk a
klasztert, akkor egy csomópontot masternek konfigurálunk.
    \item Az automatikus kijelölésnél létrehozzuk a csomópontok klaszterét és ők
választanak ki egyet maguk közül masternek.
  \end{itemize}

  \item A hátránya az inkonzisztencia.
  \begin{itemize}
    \item Az a veszély, hogy különböző kliensek különböző slave-ket
olvasva, \textbf{különböző értékeket} fognak látni, mivel nem
minden változás adódik tovább azonnal a slave-knek.
Legrosszabb esetben egy kliens nem tud olvasni egy írást,
amit most készült.
    \item Ha master-slave replikációt használunk \textbf{forró mentéshez}, ez
az eset akkor is fennáll, mivel ha a master meghibásodik,
minden olyan módosítás, amelyet nem továbbított a
mentésnek, elveszik.
  \end{itemize}
\end{itemize}


\subsubsection{Peer-to-Peer replikáció}


\begin{itemize}
  \item Minden csomópont fogadhat írást és olvasást minden adatra.
  \item Minden csomópontnak egyenlő súlya van, mindenki fogadhat írást,
és ha valamelyiküket elvesztjük, az nem akadályozza meg az
adattár elérését.
  \item A meghibásodásokat könnyen áthidalhatjuk úgy, hogy az adatokhoz
végig hozzáférünk.
  \item Könnyen adhatunk hozzá új csomópontot, hogy javítsuk a
teljesítményt.
  \item A legnagyobb bonyodalom a konzisztencia. Ha két különböző
helyet írhatunk, akkor megkockáztatjuk azt, hogy két felhasználó
ugyanazt a rekordot ugyanabban az időben próbálja módosítani,
amely írás-írás konfliktust okoz. Az olvasási inkonzisztencia is
vezethet problémákhoz, de azok legalább relatívan átmenetiek.
\end{itemize}


\subsubsection{A sharding és a replikáció kombinációja}

Ha master-slave replikációt és shardingot használunk
egyszerre, az azt jelenti, hogy több masterunk van, de egy
adatelem csak egy masteren van. Konfigurációtól függően
kiválaszthatsz egy csomópontot masternek és a többit slave-nek
vagy választhatsz több mastert és több slavet.

Peer-to-peer replikáció és sharding használata az
oszlopcsalád adatbázisoknál szokásos stratégia.

Jó kiindulási pont a peer-to-peer replikációkhoz, ha a replikációs
faktor 3, így minden shard 3 csomóponton jelenik meg. Ha egy
csomópont meghibásodik, akkor a shard-jai a többi
csomóponton felépülnek.

\subsubsection{Konzisztencia}

\textbf{Módosítási konzisztencia}:

\begin{itemize}
  \item \textbf{Írás-írás konfliktus}: két ember ugyanabban az
időpontban ugyanazt az adatelemet módosítja.
  \item Amikor az írás eléri a szervert, akkor a szerver sorbaállítja
(serialize) őket, azaz eldönti, hogy melyiket hajtja végre
elsőként majd másodikként.
  \item \textbf{Elveszett módosítás}: Martin módosítása után Péter
azonnal felülírja az adatelemet, akkor a Martin módosítása
elveszett. Konzisztenciahibának látjuk, mert Péter
módosítása azon az állapoton hajtódott végre, amire
Martin módosított, bár azt várjuk, hogy az eredeti állapoton
hajtódjon végre.
\end{itemize}

\textbf{Módosítási konzisztencia}:

\begin{itemize}
  \item A konzisztencia kezelését kétféleképp szokták
megközelíteni: pesszimista kezelés és optimista kezelés.
  \item A \textbf{pesszimista} megközelítésben megelőzzük a
konfliktusokat; az \textbf{optimista} megközelítésben
megengedjük a konfliktusokat, de észrevesszük őket és
intézkedéseket teszünk az eltávolításukra.

  \item Az írási konfliktusokhoz a szokásos \textbf{pesszimista}
megközelítés az \textbf{írási zárak} elhelyezése, azaz ha egy
értéket meg akarunk változtatni, előtte zárat kell elhelyezni
rajta. A rendszer biztosítja, hogy egyszerre csak kliens
tehet zárat egy adatelemre. Ha Martin és Péter zárat
kérne, akkor csak Martin (az első) járna sikerrel. Péter így
látná Martin írását, mielőtt eldöntené, hogy elvégzi-e a
módosítását.

  \item A szokásos \textbf{optimista} megközelítés a \textbf{feltételes módosítás},
ahol bármely kliens (amelyik módosítást végez) a közvetlenül
a módosítás előtt leteszteli, hogy a módosítandó érték
változott-e az utolsó olvasása óta. Ebben az esetben Martin
módosítása sikeres lenne, Péteré meghiúsulna. A hibából
Péter tudná, hogy újra meg kell néznie az értéket és eldönteni,
hogy akar-e módosítani.

  \item Egy másik \textbf{optimista} megközelítés az írás-írás konfliktus
kezelésére, amikor \textbf{elmentjük mindkét módosítást} és
rögzítjük, hogy konfliktusban állnak. Ez ismerős lehet a
verziókezelő rendszerekből. A következő lépésben a két
módosítást valahogyan \textbf{össze kell olvasztani} (kézzel vagy
automatikusan).

  \item A párhuzamos programozás alapvető velejárója, hogy
mérlegelni kell a biztonság (a hibák, mint a módosítási
konfliktusok kerülése) és a gyorsaság (gyorsan
válaszoljunk a kliensnek) között.
\end{itemize}


\textbf{Olvasási konzisztencia}

\begin{itemize}
  \item \textbf{Logikai konzisztencia}: biztosítja, hogy a különböző
adatelemeknek \textbf{együtt van értelmük}. A logikai
inkonzisztencia elkerülését a relációs adatbázis a
tranzakciókkal végzi. Ha Martin két módosítása egy
tranzakcióban lenne, a rendszer garantálná, hogy Péter vagy
a módosítások előtti vagy a módosítások utáni adatokat
olvasná.
  \item Természetesen nem lehet minden adatot egy aggregátumban
elhelyezni, így minden olyan módosítás, amely több
aggregátumot érint, hagy egy időablakot, amikor a kliensek
inkonzisztens olvasást tudnak végezni. Azt az időtartamot,
amikor az inkonzisztencia fennáll \textbf{inkonzisztencia ablak}nak
hívják. Egy NoSQL rendszernek lehet nagyon rövid
inkonzisztenciaablaka: Az Amazon dokumentációja azt írja,
hogy a SimpleDB inkonzisztenciaablaka általában kevesebb,
mint egy másodperc.
  \item \textbf{Másolási (replikációs) konzisztencia}: biztosítja, hogy
ugyanannak az adatelemnek ugyanaz az értéke, amikor
különböző másolatokról olvassuk.

  \begin{itemize}
    \item Képzeljük el, hogy van egy utolsó hotelszoba egy adott időpontra. A
hotel foglalási rendszere sok csomóponton fut. Martin és Cili együtt
(mert ők egy pár) szeretnék lefoglalni a szobát, de Martin Londonban
van, Cili Budapesten. Telefonon beszélgetnek. Közben Péter Tokióból
lefoglalja a szobát. Ez módosítja a másolt szoba elérhetőségét, de a
módosítás Budapestre hamarabb odaér, mint Londonba. Amikor Martin
és Cili frissítik a böngészőjüket, hogy lássák, hogy a szoba elérhető-e
még, Cili foglaltnak látja,míg Martin szabadnak.
  \end{itemize}


  \item \textbf{Egyszer majd valamikor a jövőben (eventually)
konzisztencia}: bármely időben lehetnek a csomópontok
másolási inkonzisztensek, de ha nincs több módosítás, akkor
végső soron minden csomópont ugyanarra az értékre fog
módosulni.

  \item \textbf{Lejárt (stale)}: az adat elavult (amely emlékeztet minket arra,
hogy a cache a replikáció egy másik formája)

  \item \textbf{Olvasom az írásom (read-your-writes) konzisztencia},
amely azt jelenti, hogy ha egyszer módosítottunk valamit,
akkor azt a módosítást garantáltan látni fogjuk.

  \begin{itemize}
    \item Előfordulhat a következő szituáció: az írásunkat az egyik
csomópont fogadja és a klaszteren néhány perces az
inkonzisztenciaablak. Az utolsó írásunk után frissítjük a
böngészőnket, ami egy másik csomópontra dob át, amelyhez
még nem jutott el a módosítás. Így úgy tűnhet, mintha
elveszítettük volna az írásunkat.
  \end{itemize}

  \item Az egyik módja, hogy elérjük az olvasom az írásom (read-
your-writes) konzisztenciát egy egyébként Egyszer majd
valamikor a jövőben (eventually) konzisztens rendszerben, ha
biztosítjuk a \textbf{munkamenet (session) konzisztenciát}: egy
felhasználói munkamenet olvasom az írásom (read-your-
writes) konzisztens.

  \item Több módszer létezik a munkamenetkonzisztencia
biztosítására. Az általános és gyakran a legkönnyebb módszer
a \textbf{ragadós (sticky) munkamenet}: a munkamenet egy
csomóponthoz van kötve (ezt hívják session affinity-nek is).
A hátránya az, hogy a load balancer nem tudja olyan jól
végezni a munkáját.

  \item A munkamenetkonzisztencia másik megközelítése a
verzióbélyeg (version stamps) használata. Minden
adattárbeli érintkezéskor használjuk a munkamenet által
látható utolsó időbélyeget. A szerver csomópontoknak
biztosítaniuk kell, hogy meglegyen nekik az a módosítás,
amely a megfelelő verzióbélyeget tartalmazza, mielőtt egy
kérésre válaszol.

\end{itemize}



\textbf{The CAP Theorem}

CAP: konzisztencia (consistency), elérhetőség (availability), és
partíciótolerancia (partition tolerance)

Az elmélet szerinte csak kettőt lehet megkapni egy
rendszerben.

\begin{itemize}
  \item \textbf{Az elérhetőség}: minden egyes olvasás és írás vagy sikeresen feldolgozásra kerül
vagy egy hibaüzenetet kap, hogy a műveletet nem lehet végrehajtani.
  \item A \textbf{partíciótolerancia}: Egy rendszer akkor partíció toleráns, ha a kérésre hálózati
partíció esetén is helyes választ ad (kivéve a teljes hálózat kiesésének esetét). Ha
egy rendszer nem partíció toleráns, a hálózat partíciója esetén semmilyen garanciát
nem nyújt a konzisztenciára és a rendelkezésre állásra. A hálózat partíciója annak
alapján modellezhető, hogy a hálózat egyik csomópontjából a másikba küldött
üzenetekből tetszőleges számú elveszhet. Egy nem összefüggő hálózatban a hálózat
egyik komponenséből a másikba küldött minden üzenet elveszik.

  \item \textbf{Konzisztencia}: Egy elosztott rendszer akkor konzisztens, ha bármely időpillanatban
egy adategység értékét bármely csomóponttól lekérdezve ugyanazt az értéket kapjuk.
\end{itemize}


Az egyedülálló szerver nyilvánvaló példa a \textbf{CA rendszerre}: a
rendszer konzisztens és elérhető, de nem partíciótoleráns.
Ebben a világban él a legtöbb relációs adatbázisrendszer.


A CAP theorem valódi lényege: Gyakran azt mondják, hogy az
CAP elmélet szerinte “csak kettőt kaphatunk meg a háromból”, a gyakorlatban ez úgy hangzik, hogy ha egy rendszer
partíciókból áll, mint az elosztott rendszerek, akkor
egyensúlyozni kell a konzisztencia és az elérhetőség
között. Ez nem egy bináris döntés, gyakran kis konzisztencia
mellett döntünk, hogy jobb elérhetőséget kapjunk. Az
eredmény rendszer nem lesz tökéletesen konzisztens és nem
lesz tökéletesen elérhető, de egy olyan kombinációja lesz,
amely alkalmas a rendszer szükségleteihez.

Gyakran jobb a \textbf{konzisztencia és a lappangás között
egyensúlyozására} gondolni a konzisztencia és az elérhetőség
egyensúlyozása helyett. Elosztott környezetben a konzisztenciát
úgy javíthatjuk, ha több csomópontot vonunk be az interakcióba
(azaz az írásba vagy az olvasásba), azonban minden hozzáadott
csomópont növeli az interakció válaszidejét. Az elérhetőségre
így úgy gondolhatunk, mint a lappangás olyan korlátozására,
amelyet még tolerálni tudunk, azaz ha a lappangás túl magas
lesz, feladjuk és az adatot elérhetetlenként kezeljük. Ez
illeszkedik a CAP-ban lévő definícióhoz.


\textbf{BASE}

A NoSQL követői azt mondják, hogy a relációs
tranzakciók ACID tulajdonságai helyett a NoSQL
rendszerek a BASE tulajdonságokat követik
\textbf{(Basically Available, Soft state, Eventual
consistency) - (alapvetően elérhető, lágy
állapot, egyszer majd valamikor a jövőben
konzisztens)}


\textbf{Quorum-ok}

Minél több csomópontot bevonunk egy kérésbe, annál
nagyobb az esélyünk, hogy elkerüljük az
inkonzisztenciát. Ez természetesen felveti a kérdést:
hány csomópontot kell bevonnunk, hogy erős
konzisztenciát kapjunk?


Az \textbf{írási quorum}-ot a W>N/2 egyenlőtlenséggel
fejezhetjük ki, ahol W az írásban résztvevő
csomópontok száma, N a másolatokba bevont
csomópontok száma, amelyet úgy is hívnak, hogy
\textbf{replikációs faktor}.


\textbf{Olvasási quorum}: hány csomóponttal kell kapcsolatot
létesítened ahhoz, hogy biztos legyél abban, hogy a
legfrissebb adatot kapod. Az olvasási quorum egy
kicsit komplikáltabb, mert attól függ, hogy hány
csomópont igazolta vissza az írást. Akkor van erősen
konzisztens olvasásunk, ha R + W > N, ahol R az
olvasáskor elérendő csomópontok száma, W
csomópont igazolta vissza az írást, és a replikációs
faktor N.


Ezt az egyenlőtlenséget a peer-to-peer elosztási
modell esetén tartsuk észben. Ha egy master-slave
elosztási modellünk van, akkor csak a mastert kell írni,
hogy elkerüljük az írás-írás konfliktust és hasonlóan a
masterről kell olvasni, hogy elkerüljük az olvasás-írás
konfliktust.


\section{ER Model}


\textbf{Entity-Relationship (ER) model}: Népszerű magasszintű koncepcionális modell
\textbf{ER diagram}: Az ER modellhez kapcsolódó diagramszerű jelölésmód


ER model az adatokat, mint a
következőeket írja le:

\begin{itemize}
  \item Egyedek
  \item Tulajdonságok
  \item Kapcsolatok 
\end{itemize}


\subsection{Egyed}

A valós világ tárgya (thing, teremtése, valamije)
független létezéssel

Egyed típus - egyed előfordulás



\subsection{Attribútumok}


Attribútum típus - attribútum előfordulás

Az egyedet leíró tulajdonságok


Attribútumok típusai:

\begin{itemize}
  \item Összetett - egyszerű (atomic)
  \item Egyértékű - halmazértékű (többértékű)
  \item Tárolt - származtatott
  \item NULL értékek
  \item Komplex attribútumok (összetett és halmazértékű
attribútumok tetszőlege egymásbaágyazása)
\end{itemize}

\subsubsection{Egyedtípus}

Egyedek olyan halmaza vagy kollekciója
amelyeknek ugyanolyan attribútumaik vannak

\subsubsection{Azonosító}

Attribútumok, amelyeknek az értéke egyedi egy
egyedhalmaz egyedelőfordulásaiban

\textbf{Azonosító attribútum}: Az egyediség tulajdonságnak az egyedtípus minden
egyedhalmazára fenn kell állni. Egy egyszerű (vagy egy összetett) attribútum lehet

\subsubsection{Értékhalmaz (vagy értéktartomány)}

Az egyedek attribútumaihoz rendelhető értékek
tartományát határozza meg.

\subsubsection{Kapcsolat}

Ha egy attribútum egy másik egyedtípusra
hivatkozik, akkor a hivatkozást attribútum helyett
kapcsolattal reprezentáljuk


\textbf{Kapcsolattípus} $(R): E_1, E_2, ..., E_n$
egyedtípusok közötti viszony

Az egyedtípusok egyedelőfordulásai között
létrejövő asszociációkat definiálja

\textbf{Kapcsolat előfordulás} $r_i$

\begin{itemize}
  \item Minden $r_i$ $n$ darab önállő egyeddel van
kapcsolatban $(e_1, e_2, ..., e_n)$
  \item Minden $ri$ -beli $e_j$ egyed az $E_j$ egyedhalmaz egy
tagja
\end{itemize}


\textbf{A kapcsolat foka}

\begin{itemize}
  \item A résztvevő egyedtípusok száma
  \item Bináris, ternáris
\end{itemize}


\textbf{Kapcsolat, mint attribútum}

Egy bináris (1:1 vagy 1:n) kapcsolatot
kifejezhetünk attribútumként

\textbf{Szerepkör}

A szerepkör kifejezi, hogy a résztvevő egyed
milyen szerepet játszik a kapcsolatban



\textbf{Rekurzív kapcsolat}

Egy kapcsolattípusban ugyanaz az egyedtípus
többször vesz részt különböző szerepkörökkel

A szerepkörnevet meg kell adni


\subsubsection{Bináris kapcsolattípusok
strukturális megszorításai}

\textbf{Számosság}

Azon kapcsolat előfordulások maximális számát
határozza meg, amelyben az egyed részt vehet

1:1, 1:n, n:m

\textbf{Részvételi megszorítás}

Megadja, hogy egy egyed létezése függ-e attól, hogy
kapcsolatban áll egy másik, a kapcsolattípuson
elérhető egyeddel

A kapcsolatelőfordulások minimális számát határozzák
meg


Típusai: \textbf{totális} (létezésfüggőség) és \textbf{részleges}


\subsection{Gyenge egyedtípusok}

Nincs saját azonosító attribútumuk: Azonosítása egy másik egyedtípus egyedének
meghatározásával van kapcsolatban.

\textbf{Azonosító kapcsolat}: A gyenge egyedtípust a tulajdonosával összekötő
kapcsolat


A gyenge egyedtípus mindig totális résztvevője a
kapcsolatnak


Részleges kulcs (diszkriminátor): egyértelműen
azonosítják azokat a gyenge egyedeket, amelyek
ugyanazon tulajdonos egyedhez kapcsolódnak


\begin{figure}[H]  % or [!htbp] if you’re fine with floating
  \centering
  \includegraphics[width=1\textwidth]{er_relation.png} % adjust width/height as needed
\end{figure}


\section {ER to Relational}

\begin{enumerate}
  \item Erős egyedtípusok leképezése
  \begin{itemize}
    \item Az ER séma minden E erős egyedtípusához rendeljünk
hozzá egy R relációsémát, amely tartalmazza E összes
egyszerű attribútumát. Az összetett attribútumoknak
csak az egyszerű komponenseit adjuk hozzá R
attribútumaihoz.
    \item Válasszuk E kulcs attribútumainak egyikét az R
relációséma elsődleges kulcsául. Ha az E-ből
választott kulcs összetett, akkor annak egyszerű
attribútumai együttesen fogják alkotni R elsődleges
kulcsát.
    \item R egyedreláció
  \end{itemize}

  \item Gyenge egyedtípusok leképezése
  \begin{itemize}
    \item Az ER séma minden W gyenge egyedtípusához
rendeljünk hozzá egy R relációsémát, melynek
attribútumai legyenek W összes egyszerű attribútuma
és W összetett attribútumainak egyszerű komponensei.
Továbbá adjuk hozzá R attribútumaihoz külső kulcs
attribútumként azoknak a tulajdonos relációsémáknak
az elsődleges kulcs attribútumait.
    \item R elsődleges kulcsa a tulajdonos egyedtípusok
elsődleges kulcsainak és a W gyenge egyedtípus
diszkriminátorának az együttese.
  \end{itemize}

  \item Bináris 1 : 1 számosságú
kapcsolattípusok leképezése

  \begin{itemize}
    \item Külső kulcs használata: Válasszuk ki az egyik relációt
(mondjuk S-t) és vegyük fel S külső kulcsaként T
elsődleges kulcsát. Célszerű S-nek azt a relációt
választani, amelyiket abból az egyedtípusból
képeztünk le, amelyik totális résztvevője az R
kapcsolatnak.
    \item Vegyük fel továbbá R egyszerű attribútumait, illetve R
összetett attribútumainak egyszerű komponenseit S
attribútumaiként. 
  \end{itemize}

  vagy

  \begin{itemize}
    \item Összevonás: Egy másik lehetőség az 1 : 1
kapcsolatok leképezésére, ha a két
egyedtípust és a kapcsolatot egyetlen
relációba vonjuk össze. Ezt akkor tehetjük
meg, ha mindkét egyedtípus totális résztvevő a
kapcsolatnak.
  \end{itemize}

  vagy

  \begin{itemize}
    \item Kereszthivatkozás vagy kapcsoló reláció használata: A harmadik
lehetőség, hogy felveszünk egy harmadik R relációt abból a
célból, hogy kereszthivatkozással lássuk el a két egyedtípusból
képzett S és T relációk elsődleges kulcsait.
    \item Az R reláció az S és a T elsődleges kulcsait tartalmazza, mint
külső kulcs. Az R elsődleges kulcsa S vagy T elsődleges kulcsa
lesz, a másik külső kulcsra pedig egyediségi megszorítást
tehetünk.
    \item Vegyük fel továbbá R egyszerű attribútumait, illetve R összetett
attribútumainak egyszerű komponenseit S attribútumaiként.
    \item Az R relációt kapcsoló relációnak nevezzük
  \end{itemize}


  \item Bináris 1 : N számosságú
kapcsolattípusok leképezése

  \begin{itemize}
    \item R kapcsolattípus esetén meg kell határozni azt az S
relációt, amelyiket a kapcsolattípus N-oldali
egyedtípusából képeztünk.
    \item Vegyük fel S külső kulcsaként az R-ben részt vevő
másik egyedtípusból képzett T reláció elsődleges
kulcsát.
    \item Vegyük fel továbbá R egyszerű attribútumait, illetve
R összetett attribútumainak egyszerű komponenseit
S attribútumaiként.
  \end{itemize}

  vagy

  \begin{itemize}
    \item most is használhatunk kapcsoló relációt
(kereszthivatkozást), ahogy az 1 : 1 kapcsolatoknál
tettük. Ekkor egy külön R relációt hozunk létre,
amelynek attribútumai S és T elsődleges kulcsai, és
amelynek elsődleges kulcsa megegyezik S
elsődleges kulcsával.
  \end{itemize}


  \item Bináris M : N számosságú
kapcsolattípusok leképezése

  \begin{itemize}
    \item hozzunk létre egy új S relációt, amely R-et reprezentálja.
Vegyük fel S külső kulcsaként a kapcsolatban részt vevő
egyedtípusokból képzett relációk elsődleges kulcsait; ezek
együttese alkotja S elsődleges kulcsát. Vegyük fel továbbá R
egyszerű attribútumait, illetve R összetett attribútumainak
egyszerű komponenseit S attribútumaiként.
    \item S kapcsoló reláció
  \end{itemize}

  \item Többértékű attribútumok leképezése
  
  \begin{itemize}
    \item hozzunk létre egy új R relációt.
    \item Ez az R reláció tartalmazzon egy, az A-nak megfelelő
attribútumot, valamint annak a relációnak a K elsődleges
kulcsát - R külső kulcsaként -, amelyet az A-t tartalmazó
egyedtípusból vagy kapcsolattípusból képeztünk. R elsődleges
kulcsát A és K együttese alkotja.
    \item Ha a többértékű attribútum összetett, akkor az egyszerű
komponenseit vegyük fel R attribútumaiként.
  \end{itemize}

  \item N-edfokú kapcsolattípusok
leképezése

  \begin{itemize}
    \item hozzunk létre egy új S relációt, amely R-et reprezentálja.
    \item Vegyük fel S külső kulcsaként a kapcsolatban részt vevő
egyedtípusokból képzett relációk elsődleges kulcsait. Vegyük
fel továbbá R egyszerű attribútumait, illetve R összetett
attribútumainak egyszerű komponenseit S attribútumaiként.
    \item S elsődleges kulcsa általában az összes külső kulcs
együttese.
  \end{itemize}
\end{enumerate}

\section{EER Model}


EER modell minden ER modellbeli fogalmat
tartalmaz, es azokon felül még tartalmazza a
következő fogalmakat:


\begin{itemize}
  \item Alosztály és szuperosztály
  \item Specializáció és generalizáció
  \item Kategória vagy uniótípus
  \item Attribútum és kapcsolat öröklődés
\end{itemize}


\subsection{Alosztály, szuperosztály, és
öröklődés}

Egy egyedtípus altípusa vagy alosztálya egyedek egy alcsoportja, amelynek van
jelentősége.

Az adatbázis-alkalmazásra vonatkozó
jelentőségük miatt külön kell őket reprezentálni

Például:

Secretary, Technician, Enginner, Manager -
alosztályai az Employee szuperosztálynak

\subsubsection{Típus öröklődés}

Az alosztályok egyedei öröklik a szuperosztály
minden attribútumát és kapcsolatát

\begin{figure}[H]  % or [!htbp] if you’re fine with floating
  \centering
  \includegraphics[width=1\textwidth]{eer_model.png} % adjust width/height as needed
\end{figure}

\subsection{Specializáció}

Az a folyamat, amely egy egyedtípus alosztályait definiálja.

Az alosztályok definiálása a szuperosztályban lévő egyedek
bizonyos jellemzői alapján történik.

Az EER diagramon egy specializációs körrel jelölik.

Az alosztályokat lehet meghatározni:

\begin{itemize}
  \item Attribútum(ok) alapján
  \item Vagy kapcsolattípus alapján
\end{itemize}

Példa:

Secretary, Technician, stb. - alosztályai az Employee
szuperosztálynak, a meghatározó attribútum a job.

Hourly\_employee alosztálya az Employee szuperosztálynak, a
meghatározó kapcsolattípus a Belongs\_to

\subsubsection{Generalizáció}

Az a folyamat, amelyben adott egyedtípusokból
egy általános egyedtípust határozunk meg.

A specializáció ellentétes folyamata

\textbf{Általánosítás} egy szuperosztályba: Az eredeti egyedtípusok speciális alosztályok

Lehet egy vagy több alosztály

Egyed altípus meghatározása:

\begin{itemize}
  \item \textbf{Predikátumdefiniált} (vagy feltételdefiniált)
alosztályok (predikátum: job\_type='Secretary') (az
EER diagramon az alosztály és a spec. kör közötti
vonal mellé írjuk)
  \item \textbf{Attribútumdefiniált} specializáció (az EER diagramon
a spec. kör mellé írjuk az attribútum nevét, az értékét
az alosztály mellé)

  \item Felhasználó által definiált (a felhasználó az egyes
egyedekről egyesével eldönti, hogy melyik alosztályba
tartoznak)
\end{itemize}

\subsubsection{A specializáció és generalizáció
megszorításai}


\textbf{Diszjunkt megszorítás}

Meghatározza, hogy a specializáció altípusai
diszjunktak (azaz egy egyed csak egy altípus
eleme lehet).

A spec. körbe d-t írunk.

Ha az alosztályok nem diszjunktak, akkor az
egyedeik halmaza átfedő, o-val jelöljük a
körben.

\textbf{Teljesség megszorítás}

Lehet teljes vagy részleges

\textbf{Teljes specializáció}: a szupertípus minden
egyedének legalább egy alosztályban
szerepelnie kell (dupla vonal)

\textbf{Részleges specializáció}: enged olyan
egyedeket, amelyek egyik alosztályhoz sem
tartoznak.


A diszjunkság és a teljesség független
egymástól

\subsubsection{Uniótípus modellezése
kategóriák használatával}

\textbf{Uniótípus vagy kategória}

\begin{itemize}
  \item Egy szuperosztály/alosztály kapcsolatot képvisel, ahol
egynél több szuperosztály van
  \item Az alosztály reprezentálja az objektumok kollekcióját,
amely a különböző szuperosztályok uniójának
részhalmaza
  \item Az attribútumöröklődés sokkal szelektívebben működik
  \item A kategória lehet totális vagy részleges
  \item A spec. körben egy u jelöli
\end{itemize}

\begin{figure}[H]  % or [!htbp] if you’re fine with floating
  \centering
  \includegraphics[width=1\textwidth]{union_type.png} % adjust width/height as needed
\end{figure}


\section{EER modell leképezése
relációs modellre}

\begin{enumerate}
  \item Specializációk és generalizációk
leképezésének lehetőségei


  \begin{itemize}
    \item 8A: Több reláció - szuperosztály és
    alosztályok
    
    \begin{itemize}
      \item Ez a lehetőség mindenféle specializáció esetén
(totális vagy részleges, diszjunkt vagy átfedő)
működik.
      \item Hozzunk létre egy relációt a C szuperosztály
számára C attribútumaival, az elsődleges kulcsa k
legyen.
      \item Hozzunk létre egy relációt minden alosztályhoz $S_i$
$(1 \le i \le m)$ az $S_i$ attribútumaival és k-val. Az
elsődleges kulcs k.
    \end{itemize}




    \item 8B: Több reláció - csak alosztály relációk
    
    \begin{itemize}
      \item Ez a lehetőség csak olyan specializáció esetén
működik, ahol az alosztályok totálisak. Ha a
specializáció átfedő, egy egyed több relációban is
felbukkanhat, emiatt diszjunkt alosztályoknál
javasolt.
      \item Hozzunk létre egy $S_i$ relációt minden alosztályhoz
az $S_i$ és a $C$ attribútumaival. Az elsődleges kulcs a
k.
    \end{itemize}




    \item 8C: Egyetlen reláció egy típusattribútummal
    

    \begin{itemize}
      \item Ez a lehetőség csak olyan specializáció esetén működik,
amely diszjunkt.

      \item Fennáll a veszélye annak, hogy ez a megoldás sok
NULL értéket generál, ha sok speciális attribútum
szerepel az alosztályban.

      \item Hozzunk létre egyetlen relációt a következő
attribútumokkal:
$$\{k, a_1, … , a_n\} \cup \{S_1 \text{attribútumai}\} \cup ... \cup
\{S_m \text{attribútumai}\} \cup \{t\}$$ Az elsődleges kulcs a k

      \item A t-t típus (vagy diszkrimináló) attribútumnak nevezzük,
amely jelzi azt az alosztályt, amelyhez az egyes
rekordok tartoznak.
    \end{itemize}




    \item 8D: Egyetlen reláció több típusattribútummal
    

    \begin{itemize}
      \item Ez a lehetőség olyan specializációk esetén is működik,
amely átfedő alosztályokat tartalmaz.

      \item Diszjunkt specializációra is megfelelő
      
      \item Hozzunk létre egy relációt a következő attribútumokkal:
$$\{k, a_1, ... , a_n\} \cup \{S_1 attribútumai\} \cup ... \cup
\{S_m attribútumai\} \cup \{t_1, t_2, ..., t_m\}$$. Az elsődleges kulcs k.
    
      \item Minden $t_i (1 \le i \le m)$ egy logikai típusú attribútum, amely
azt jelzi, hogy egy adott rekord az $S_i$ osztályhoz tartozik-e.
\end{itemize}





  \end{itemize}

  \item Kategóriák (uniótípusok) leképezése
  
  \begin{itemize}
    \item Különböző kulcsokkal rendelkező szuperosztályok által
definiált kategória leképezéséhez célszerű egy új
kulcsattribútumot bevezetni, amelyet helyettesítő
kulcsnak nevezünk a kategóriának megfelelő reláció
létrehozásakor.

    \item A helyettesítő kulcs attribútumot minden olyan
relációba is felvesszük, amelyeket a kategória
szuperosztályaiból képezünk.
  \end{itemize}

\end{enumerate}



\section{Normal Forms}

\subsection{A funkcionális függés definíciója}


Az $R$ relációséma két attribútumhalmaza, $X$ és $Y$ között,
$X \rightarrow Y$-nal jelölt funkcionális függés előír egy megszorítást
azokra a lehetséges rekordokra, amelyek az $R$ egy $r$
relációs állapotát valósíthatják meg. A megszorítás az,
hogy bármely két, $r$-beli $t_1$ és $t_2$ rekord esetén, amelyekre
$t_1[X] = t_2[X]$ teljesül, teljesülnie kell $t_1[Y] = t_2[Y]$-nak is.

Az attribútumok szemantikájának vagy jelentésének a
tulajdonsága

Minden lehetséges relációs állapotnak meg kell felelnie a
funkcionális függés megszorításnak


\subsection{A funkcionális függés tulajdonságai}


\begin{itemize}
    \item \textbf{Reflexív:} $X \supseteq Y \Rightarrow X \to Y$
    \item \textbf{Augmentív:} $X \to Y \Rightarrow XZ \to YZ$
    \item \textbf{Tranzitív:} $X \to Y, Y \to Z \Rightarrow X \to Z$
    \item \textbf{Dekompozíciós tulajdonság:}\\
    $X \to YZ \Rightarrow X \to Y$
    \item \textbf{Additív:} $X \to Y, X \to Z \Rightarrow X \to YZ$
    \item \textbf{Pszeudotranzitív:} $X \to Y, WY \to Z \Rightarrow WX \to Z$
\end{itemize}


\subsection{Lezárt}

F: egy R relációs sémán meghatározott
funkcionális függések halmaza.

F lezártja funkcionális függések halmaza,
amely minden olyan funkcionális függést
tartalmaz, amely F-ből következtethető.

Jele: F+

\subsection{Armstrong axióma}

A funkcionális függés reflexív, augmentív és tanzitív
szabálya együtt helyes és teljes.

Helyes: ha adott egy R relációsémán fennálló funkcionális
függéseknek egy F halmaza, akkor bármilyen függés,
amely levezethető F-ből a három szabály segítségével,
fenn fog állni R minden olyan r relációjában, amely
kielégíti az F-beli függéseket.

Teljes: az F+ (F lezártja) halmaz meghatározható F-ből a
3 szabály alkalmazásával.

A reflexivitás, az augmentivitás és a tranzitivitás
szabályait együtt Armstrong-axiómáknak nevezzük.


\subsection{Első Normálforma}

A relációs modellbeli reláció formális definíciójának a
része


Csak atomi (vagy oszthatatlan) értékek szerepelhetnek az
1NF-ben (azaz nem szerepelhetnek összetett és
halmazértékű attribútumok)

Feltételezzük, hogy a relációnak van elsődleges kulcsa.


Hogyan érjük el az első normálformát?

\begin{itemize}
  \item Távolítsuk el az attribútumot egy másik relációba
  \item (Bővítsük a kulcsot)
  \item Használjunk több atomi attribútumot
\end{itemize}


\subsection{Részleges függés}

Egy $X \rightarrow Y$ funkcionális függés részleges
függés, ha X valamely A attribútumait
eltávolítva X-ből a függés még fennáll.


Egy $X \rightarrow Y$ funkcionális függés teljes
funkcionális függés, ha X bármely A
attribútumát eltávolítva X-ből a függés már
nem áll fenn.

\subsection{Második Normálforma}

Egy R relációséma második
normálformában (2NF-ben) van, ha első
normálformában van és R minden nem
elsőrendű (leíró) attribútuma teljesen
funkcionálisan függ R elsődleges kulcsától
(azaz nem tartalmaz részleges függést).


2NF-re alakítás:

\begin{itemize}
  \item Az eredeti relációból eltávolítjuk a
részelegesen függő nem elsőrendű (leíró)
attribútumot (A) egy másik relációba. Ebben a
második relációban szerepelnie kell az eredeti
reláció elsődleges kulcsának azon részének
(B), amelytől a nem elsőrendű attribútum függ.
A második reláció elsődleges kulcsa B, azaz az
eredeti reláció elsődleges kulcsának része.
\end{itemize}


\begin{figure}[H]  % or [!htbp] if you’re fine with floating
  \centering
  \includegraphics[width=1\textwidth]{2NF.png} % adjust width/height as needed
\end{figure}


\subsection{Tranzitív függés}

Egy R relációséma $X \rightarrow Y$ funkcionális
függése tranzitív függés, ha létezik egy
olyan Z attribútumhalmaz R-ben, amely
nem kulcsjelölt és nem része R egyetlen
kulcsának sem, és fennáll $X \rightarrow Z $ és $Z \rightarrow Y$.


\subsection{Harmadik Normálforma}


Egy R relációséma harmadik
normálformában (3NF-ben) van, ha
második normálformában (2NF-ben) van,
és nincs R-nek olyan nem elsőrendű (leíró)
attribútuma, amely tranzitívan függne az
elsődleges kulcstól.


3NF-re alakítás:

Az eredeti relációból eltávolítjuk a tranzitívan
függő nem elsőrendű (leíró) attribútumot egy
másik relációba. Ebben a második relációban
elsődleges kulcsként kell szerepelnie azoknbak
az attribútumoknak, amelyektől a nem
elsőrendű attribútumok függenek.

\begin{figure}[H]  % or [!htbp] if you’re fine with floating
  \centering
  \includegraphics[width=1\textwidth]{3NF.png} % adjust width/height as needed
\end{figure}


\subsection{Boyce-Codd Normálforma}


Egy R relációséma Boyce-Codd-féle
normálformában (BCNF-ben) van, ha
valahányszor egy $X \rightarrow A$ nemtriviális
funkcionális függés fennáll R-en, akkor X
egy szuperkulcsa R-nek.

Minden BCNF-ben lévő reláció 3NF-ben is van, viszont a 3NF-ben lévő relációk nem szükségszerűen
vannak BCNF-ben

3NF-ben A lehet elsőrangú, BCNF-ben nem



\subsection{Többértékű függés}

Az 1NF következménye
(kettő vagy több halmazértékű attribútum
esetén kibővítjük az elsődleges kulcsot)


Egy $R$ relációsémán megadott $X \twoheadrightarrow Y$ többértékű függés (ahol $X$ és $Y$ $R$ attribútumhalmazai) a következő megszorítást jelenti $R$ bármely $r$ relációs állapotára vonatkozóan: 

Ha van két olyan $t_1$ és $t_2$ rekord $r$-ben, amelyre $t_1[X] = t_2[X]$, akkor léteznie kell két $t_3$ és $t_4$ rekordnak is $r$-ben a következő tulajdonságokkal, ahol $Z$-t az $(R - (X \cup Y))$ jelölésére használjuk:

\begin{itemize}
    \item $t_3[X] = t_4[X] = t_1[X] = t_2[X]$.
    \item $t_3[Y] = t_1[Y]$ és $t_4[Y] = t_2[Y]$.
    \item $t_3[Z] = t_2[Z]$ és $t_4[Z] = t_1[Z]$.
\end{itemize}

\noindent ($X \twoheadrightarrow Y$: $X$ többértékűen meghatározza $Y$-t)


\subsection{Negyedik normálforma (4NF)}


Feltételei sérülnek, ha egy relációban
nemkívánatos többértékű függés van.

Egy R relációséma negyedik normálformában
(4NF-ben) van, figyelembe véve az F függések
halmazát (amely magában foglalja a
funkcionális és többértékű függéseket), ha
minden F+-beli nemtriviális X ->>Y többértékű
függés esetén X szuperkulcsa R-nek.


Ha egy reláció egy nemtriviális többértékű
függés miatt nincs 4NF-ben, akkor bontsuk
fel 4NF-ben lévő relációk egy halmazára.

\begin{figure}[H]  % or [!htbp] if you’re fine with floating
  \centering
  \includegraphics[width=1\textwidth]{4NF.png} % adjust width/height as needed
\end{figure}


\section{UML}

\subsection{Interaction models}

All systems involve interaction of some kind.

\begin{itemize}
  \item user interaction, which involves user inputs and
outputs: it helps to identify user requirements.
  \item interaction between the system being developed and
other systems,
  \item  interaction between the components of the system.
\end{itemize}

\subsubsection{Use case modeling}

Use case models and sequence diagrams present interaction at
different levels of detail and so may be used together.

A use case can be taken as a simple scenario that describes what a
user expects from a system.

Each use case represents a discrete task that involves external
interaction with a system.

In its simplest form, a use case is shown as an ellipse with the
actors involved in the use case represented as stick figures.

Use case diagrams give a fairly simple overview of an interaction so
you have to provide more detail to understand what is involved.
This detail can either be a simple textual description, a structured
description in a table, or a sequence diagram.

\begin{figure}[H]  % or [!htbp] if you’re fine with floating
  \centering
  \includegraphics[width=1\textwidth]{use_case_modeling.png} % adjust width/height as needed
\end{figure}

\subsubsection{Sequence diagrams}

Primarily used to model the interactions between
the actors and the objects in a system and the
interactions between the objects themselves.

The UML has a rich syntax for sequence diagrams,
which allows many different kinds of interaction
to be modeled.

It shows the sequence of interactions that take
place during a particular use case or use case
instance.

\begin{figure}[H]  % or [!htbp] if you’re fine with floating
  \centering
  \includegraphics[width=1\textwidth]{use_case_modeling.png} % adjust width/height as needed
\end{figure}


\begin{itemize}
  \item The objects and actors involved are listed along the top of
the diagram, with a dotted line drawn vertically from these.

  \item Interactions between objects are indicated by annotated
arrows. 

  \item The rectangle on the dotted lines indicates the lifeline of
the object concerned.

  \item You read the sequence of interactions from top to bottom.
  
  \item The annotations on the arrows indicate the calls to the
objects, their parameters, and the return values.
\end{itemize}


Structural models of software display the organization
of a system in terms of the components that make up
that system and their relationships.

\subsubsection{Class Diagrams}

Class diagrams are used when developing an object-
oriented system model to show the classes in a system
and the associations between these classes.

An object class can be thought of as a general
definition of one kind of system object.

An association is a link between classes that indicates
that there is a relationship between these classes.


Each class may have to have some knowledge of its
associated class.


\begin{figure}[H]  % or [!htbp] if you’re fine with floating
  \centering
  \includegraphics[width=1\textwidth]{class_diagram.png} % adjust width/height as needed
\end{figure}

At this level of detail, class diagrams look like
semantic data models.
Semantic data models are used in database
design.
They show the data entities, their associated
attributes, and the relations between these
entities.


\begin{figure}[H]  % or [!htbp] if you’re fine with floating
  \centering
  \includegraphics[height=0.4\textwidth]{class_diagram2.png} % adjust width/height as needed
\end{figure}


In the UML, you show attributes and
operations by extending the simple
rectangle that represents a class:

\begin{itemize}
  \item The name of the object class is in the
top section.

  \item The class attributes are in the middle
section. This must include the attribute
names and, optionally, their types.

  \item The operations (called methods in Java
and other OO programming languages)
associated with the object class are in
the lower section of the rectangle.
\end{itemize}



\begin{figure}[H]  % or [!htbp] if you’re fine with floating
  \centering
  \includegraphics[width=1\textwidth]{class_diagram3.png} % adjust width/height as needed
\end{figure}


\textbf{Generalization} is used to manage complexity. Rather
than learn the detailed characteristics of every entity
that we experience, we place these entities in more
general classes (animals, cars, houses, etc.) and learn
the characteristics of these classes. This allows us to
infer that different members of these classes have
some common characteristics (e.g., squirrels and rats
are rodents). We can make general statements that
apply to all class members (e.g., all rodents have teeth
for gnawing).

In modeling systems, it is often useful to examine the
classes in a system to see if there is scope for
generalization.


This means that common information will be maintained in
one place only.


If changes are proposed, then you do not have to look at all
classes in the system to see if they are affected by the
change.

In object-oriented languages, such as Java, generalization is
implemented using the class inheritance mechanisms built
into the language.

\begin{figure}[H]  % or [!htbp] if you’re fine with floating
  \centering
  \includegraphics[width=1\textwidth]{class_diagram4.png} % adjust width/height as needed
\end{figure}

\textbf{Aggregation}

\begin{figure}[H]  % or [!htbp] if you’re fine with floating
  \centering
  \includegraphics[width=1\textwidth]{aggregation.png} % adjust width/height as needed
\end{figure}


Objects in the real world are often composed of
different parts.

Aggregation means that one object (the whole) is
composed of other objects (the parts).


To show this, we use a diamond shape next to the
class that represents the whole.



\subsubsection{Behavioral models}

\textbf{Event-driven modeling}

It shows how a system responds to external and internal
events.
It is based on the assumption that a system has a finite
number of states and that events (stimuli) may cause a
transition from one state to another.

It is particularly appropriate for real-time systems.

The UML supports event-based modeling using state
diagrams, which were based on Statecharts. State diagrams
show system states and events that cause transitions from
one state to another. They do not show the flow of data
within the system but may include additional information
on the computations carried out in each state.

In UML state diagrams, rounded rectangles
represent system states. They may include a
brief description (following 'do') of the actions
taken in that state. The labeled arrows
represent stimuli that force a transition from
one state to another. You can indicate start
and end states using filled circles, as in activity
diagrams.

\begin{figure}[H]  % or [!htbp] if you’re fine with floating
  \centering
  \includegraphics[width=1\textwidth]{event_driven_modelling.png} % adjust width/height as needed
\end{figure}


For large system models, you need to hide
detail in the models. One way to do this is by
using the notion of a superstate that
encapsulates a number of separate states.
This superstate looks like a single state on a
high-level model but is then expanded to
show more detail on a separate diagram.



\section{Query Optimization}

\subsection{SQL Processing}

\begin{figure}[H]  % or [!htbp] if you’re fine with floating
  \centering
  \includegraphics[height=0.5\textwidth]{sql_processing.png} % adjust width/height as needed
\end{figure}

\subsubsection{SQL Parsing}


The parsing stage involves separating the pieces of a SQL statement into a data structure that other routines can process.
The database parses a statement when instructed by the application, which means that only the
application, and not the database itself, can reduce the number of parses.
When an application issues a SQL statement, the application makes a parse call to the database to prepare the statement for execution.
The parse call opens or creates a cursor, which is a handle for the session-specific private SQL area that holds a parsed SQL statement and other processing information. The cursor and private SQL area are in the program global area (PGA).
During the parse call, the database performs checks that identify the errors that can be found before statement execution. Some errors cannot be caught by parsing. For example, the database can encounter deadlocks or errors in data conversion only during statement execution. 


\textbf{Sytax Check}

Oracle Database must check each SQL
statement for syntactic validity. A statement that breaks a rule for well-
formed SQL syntax fails the check.

\textbf{Semantic Check}

The semantics of a statement are its meaning. A semantic check determines whether a statement is meaningful, for example, whether the objects and columns in the statement exist. A syntactically correct statement can fail a semantic check (nonexistent\_table):
SQL> SELECT * FROM nonexistent\_table;

\textbf{Shared Pool Check}


During the parse, the database performs a shared pool check to determine whether it can skip resource-intensive steps of statement processing.
To this end, the database uses a hashing algorithm to generate a hash value for every SQL statement. The statement hash value is the SQL ID shown in V\$SQL.SQL\_ID.
When a user submits a SQL statement, the database searches the shared SQL area to see if an existing parsed statement has the same hash value.


Parse operations fall into the following categories, depending on the type of statement submitted and the result of the hash check:

\begin{itemize}
  \item \textbf{Hard parse}: If Oracle Database cannot reuse existing code, then it must build a new executable version of the application code. This operation is known as a hard parse, or a library cache miss. During the hard parse, the database accesses the library cache and data dictionary cache numerous times to check the data dictionary.
  \item \textbf{Soft parse}: A soft parse is any parse that is not a hard parse. If the submitted statement is the same as a reusable SQL statement in the shared pool, then Oracle Database reuses. 
\end{itemize}

\subsubsection{SQL Optimization}

During optimization, Oracle
Database must perform a hard
parse at least once for every
unique DML statement and
performs the optimization
during this parse. The database
does not optimize DDL. The
only exception is when the DDL
includes a DML component
such as a subquery that
requires optimization.



\subsubsection{SQL Row Source Generation}

The row source generator is software
that receives the optimal execution
plan from the optimizer and produces an iterative execution plan
that is usable by the rest of the
database.

The iterative plan is a binary program that, when executed by the SQL
engine, produces the result set. The
plan takes the form of a combination of steps. Each step returns a row set.
The next step either uses the rows in this set, or the last step returns the
rows to the application issuing the
SQL statement.


A \textbf{row source} is a row set returned by a step in the execution plan along with a control structure that can iteratively process the rows. The row source can be a table, view, or result of a join or grouping operation.
The row source generator produces a row source tree, which is a collection of row sources. The row source tree shows the following information:

\begin{itemize}
  \item An ordering of the tables referenced by the statement
  \item An access method for each table mentioned in the statement
  \item A join method for tables affected by join operations in the statement
  \item Data operations such as filter, sort, or aggregation
\end{itemize}


The \textbf{execution plan} for the statement is the output of the row source generator.

\subsubsection{SQL Execution}

During
execution, the SQL engine
executes each row source in
the tree produced by the
row source generator. This
step is the only mandatory
step in DML processing.

In general, the order of the steps in execution is the reverse of the order in the plan, so you read the plan from the bottom up. Each step in an execution plan has an ID number. The numbers correspond to the Id column in the plan.
Initial spaces in the Operation column of the plan indicate hierarchical relationships. For example, if the name of an operation is preceded by two spaces, then this operation is a child of an operation préceded by one space. Operations preceded by one space are children of the SELECT statement itself.


The SQL engine executes the plan as follows:

\begin{itemize}
  \item Step 6 uses a full table scan to retrieve all rows from the departments table.
  \item Step 5 uses a full table scan to retrieve all rows from the jobs table.
  \item Step 4 scans the emp\_name\_ix index in order, looking for each key that begins with the letter A and retrieving the corresponding rowid.
  \item Step 3 retrieves from the employees table the rows whose rowids were returned by Step 4.
  \item Step 2 performs a hash join, accepting row sources from Steps 3 and 5, joining each row from the Step 5 row source to its corresponding row in Step 3, and returning the resulting rows to Step 1.
  \item Step 1 performs another hash join, accepting row sources from Steps 2 and 6, joining each row from the Step 6 source to its corresponding row in Step 2, and returning the result to the client.
\end{itemize}


During execution, the database reads the data from
disk into \textbf{memory} if the data is not in memory. The
database also takes out any \textbf{locks} and latches
necessary to ensure data integrity and logs any
changes made during the SQL execution. The final
stage of processing a SQL statement is \textbf{closing the
cursor}.

\subsection{Query Optimizer Concepts}

The query optimizer (called simply the optimizer) is built-in database software that determines the most efficient method for a SQL statement to access requested data.
Purpose of the Query Optimizer: The optimizer attempts to generate the most optimal execution plan for a SQL statement. The optimizer choose the plan with the lowest cost among all considered candidate plans. The optimizer uses available statistics to calculate cost. For a specific query in a given environment, the cost computation accounts for factors of query execution such as I/Ó, CPU, and communication.
For example, a query might request information about employees who are managers. If the optimizer statistics indicate that 80\% of employees are managers, then the optimizer may decide that a full table scan is most efficient. However, if statistics indicate that very few employees are managers, then reading an index followed by a table access by rowid may be more efficient than a full table scan.
Because the database has many internal statistics and tools at its disposal, the optimizer is usually in a better position than the user to determine the optimal method of statement execution. For this reason, all SQL statements use the optimizer.


\subsubsection{Cost-Based Optimization}



Query optimization is the process of choosing the most efficient
means of executing a SQL statement.
SQL is a nonprocedural language, so the optimizer is free to merge,
reorganize, and process in any order. The database optimizes each SQL statement based on statistics collected about the accessed
data. The optimizer determines the optimal plan for a SQL
statement by examining multiple access methods, such as full table
scan or index scans, different join methods such as nested loops
and hash joins, different join orders, and possible transformations.
For a given query and environment, the optimizer assigns a relative numerical cost to each step of a possible plan, and then factors
these values together to generate an overall cost estimate for the
plan. After calculating the costs of alternative plans, the optimizer
chooses the plan with the lowest cost estimate. For this reason,
the optimizer is sometimes called the cost-based optimizer (CBO) to
contrast it with the legacy rule-based optimizer (RBO).


\subsubsection{Execution Plans}


An execution plan describes a \textbf{recommended method of
execution for a SQL statement}.
The plan shows the \textbf{combination of the steps} Oracle
Database uses to execute a SQL statement. Each step
either retrieves rows of data physically from the database
or prepares them for the user issuing the statement.
An execution plan \textbf{displays the cost} of the entire plan,
indicated on line 0, and each separate operation. \textbf{The
cost is an internal unit that the execution plan only
displays to allow for plan comparisons}. Thus, you cannot
tune or change the cost value.

\textbf{Query Subplans}

For each query block, the optimizer generates a query subplan.
The database optimizes query blocks separately from the bottom up. Thus, the database optimizes the innermost query block first and generates a subplan for it, and then generates the outer query block representing the entire query.
The number of possible plans for a query block is proportional to the number of objects in the FROM clause. This number rises exponentially with the number of objects.
For example, the possible plans for a join of five tables are significantly higher than the possible plans for a join of two tables.

\subsection{Optimizer Components}

\begin{figure}[H]  % or [!htbp] if you’re fine with floating
  \centering
  \includegraphics[height=0.5\textwidth]{optimizer_components.png} % adjust width/height as needed
\end{figure}

\begin{itemize}
  \item \textbf{Query Transformer}: The optimizer determines
whether it is helpful to change the form of the
query so that the optimizer can generate a better
execution plan.
  \item \textbf{Estimator}: The optimizer estimates the cost of each
plan based on statistics in the data dictionary.


  \item \textbf{Plan Generator}: The optimizer compares the costs
of plans and chooses the lowest-cost plan, known
as the execution plan, to pass to the row source
generator.
\end{itemize}

\subsubsection{Query Transformer}

For some statements, the query transformer
determines whether it is advantageous to rewrite the
original SQL statement into a semantically equivalent
SQL statement with a lower cost. When a viable
alternative exists, the database calculates the cost of
the alternatives separately and chooses the lowest-
cost alternative.

\subsubsection{Query Transformer}

An example: the query
transformer rewriting an
input query that uses OR
into an output query that
uses UNION ALL.

\begin{figure}[H]  % or [!htbp] if you’re fine with floating
  \centering
  \includegraphics[height=0.5\textwidth]{query_transformer.png} % adjust width/height as needed
\end{figure}

\subsubsection{Estimator}


The estimator is the component of the optimizer that
determines the overall cost of a given execution plan.
The estimator uses three different measures to
determine cost:

\begin{itemize}
  \item \textbf{Selectivity}: The percentage of rows in the row set
that the query selects, with 0 meaning no rows and 1
meaning all rows. Selectivity is tied to a query
predicate, such as WHERE last\_name LIKE 'A\%', or a
combination of predicates. A predicate becomes
more selective as the selectivity value approaches 0
and less selective (or more unselective) as the value
approaches 1.
  \item \textbf{Cardinality}: The cardinality is the number of rows
returned by each operation in an execution plan. This
input, which is crucial to obtaining an optimal plan, is
common to all cost functions. The estimator can derive
cardinality from the table statistics collected by
DBMS\_STATS, or derive it after accounting for effects
from predicates (filter, join, and so on), DISTINCT or
GROUP BY operations, and so on. The Rows column in an
execution plan shows the estimated cardinality.
  \item \textbf{Cost}: This measure represents units of work or
resource used. The query optimizer uses disk I/O, CPU
usage, and memory usage as units of work.
\end{itemize}

If statistics are available, then the estimator uses
them to compute the measures. The statistics
improve the degree of accuracy of the measures.

\subsubsection{Plan Generator}


The plan generator explores various plans for a query block by trying out different access paths, join methods, and join orders.
Many plans are possible because of the various combinations that the database can use to produce the same result. The optimizer picks the plan with the lowest cost.
The optimizer uses an internal cutoff to reduce the number of plans it tries when finding the lowest-cost plan. The cutoff is based on the cost of the current best plan. If the current best cost is large, then the optimizer explores alternative plans to find a lower cost plan. If the current best cost is small, then the optimizer ends the search swiftly because further cost improvement is not significant.


\subsection{Query Transformations}

It describes the most important optimizer techniques
for transforming queries.
The optimizer decides whether to use an available
transformation based on cost. Transformations may
not be available to the optimizer for a variety of
reasons, including hints or lack of constraints.
For example, transformations such as subquery
unnesting are not available for hybrid partitioned
tables, which contain external partitions that do not
support constraints.


\subsubsection{OR Expansion}

In OR expansion, the optimizer transforms a query block containing
top-level disjunctions into the form of a UNION ALL query that
contains two or more branches.
The optimizer achieves this goal by splitting the disjunction into its components, and then associating each component with a branch
of a UNION ALL query.
The optimizer can choose OR expansion for various reasons. For
example, it may enable more efficient access paths or alternative join methods that avoid Cartesian products. As always, the
optimizer performs the expansion only if the cost of the
transformed statement is lower than the cost of the original
statement.
In previous releases, the optimizer used the CONCATENATION
operator to perform the OR expansion. Starting in Oracle Database 12c Release 2 (12.2), the optimizer uses the UNION-ALL operator
instead.


SELECT * FROM employees e, departments d
WHERE (e.email='SSTILES' OR
d.department\_name='Treasury')
AND e.department\_id = d.department\_id;


SELECT * FROM employees e, departments d
WHERE e.email = 'SSTILES'
AND e.department\_id = d.department\_id
UNION ALL
SELECT * FROM employees e, departments d
WHERE d.department\_name = 'Treasury'
AND e.department\_id = d.department\_id;

\subsubsection{View Merging}

In view merging, the optimizer merges the query block representing
a view into the query block that contains it.
View merging can improve plans by enabling the optimizer to
consider additional join orders, access methods, and other transformations. For example, after a view has been merged and
several tables reside in one query block, a table inside a view may
permit the optimizer to use join elimination to remove a table outside the view.
For certain simple views in which merging always leads to a better
plan, the optimizer automatically merges the view without
considering cost. Otherwise, the optimizer uses cost to make the determination. The optimizer may choose not to merge a view for
many reasons, including cost or validity restrictions (privileges).
You can use hints to override view merging rejected because of cost 


\noindent In \textbf{simple view merging}, the optimizer merges select-project-join views. 
For example, a query of the employees table contains a subquery that joins the departments and locations tables.

\vspace{0.5em}

\noindent Simple view merging frequently results in a more optimal plan because of the additional join orders and access paths available after the merge.

\vspace{0.5em}

\noindent A view may not be valid for simple view merging because:
\begin{itemize}
    \item The view contains constructs not included in select-project-join views, including: GROUP BY, DISTINCT, Outer join, MODEL, CONNECT BY, Set operators, Aggregation
    \item The view appears on the right side of a semijoin (Exists, in) or antijoin (not exist, not in).
    \item The view contains subqueries in the SELECT list.
    \item The outer query block contains PL/SQL functions.
    \item The view participates in an outer join, and does not meet one of the several additional validity requirements that determine whether the view can be merged.
\end{itemize}



\noindent The following query joins the \texttt{hr.employees} table with the \texttt{dept\_locs\_v} view, which returns the street address for each department. \texttt{dept\_locs\_v} is a join of the \texttt{departments} and \texttt{locations} tables.

\begin{verbatim}
SELECT e.first_name, e.last_name, dept_locs_v.street_address, dept_locs_v.postal_code
FROM employees e, (SELECT d.department_id, d.department_name, l.street_address, l.postal_code
   FROM departments d, locations l
   WHERE d.location_id = l.location_id ) dept_locs_v
WHERE dept_locs_v.department_id = e.department_id AND e.last_name = 'Smith';
\end{verbatim}

\noindent The database can execute the preceding query by joining \texttt{departments} and \texttt{locations} to generate the rows of the view, and then joining this result to \texttt{employees}. Because the query contains the view \texttt{dept\_locs\_v}, and this view contains two tables, the optimizer must use one of the following join orders:

\begin{itemize}
    \item employees, \texttt{dept\_locs\_v} (departments, locations)
    \item employees, \texttt{dept\_locs\_v} (locations, departments)
    \item \texttt{dept\_locs\_v} (departments, locations), employees
    \item \texttt{dept\_locs\_v} (locations, departments), employees
\end{itemize}

\noindent Join methods are also constrained. The index-based nested loops join is not feasible for join orders that begin with \texttt{employees} because no index exists on the column from this view. Without view merging, the optimizer generates the following execution plan:

\textbf{Complex View Merging}

\noindent The transformed query is cheaper than the untransformed query, so the optimizer chooses to merge the view. In the untransformed query, the \textbf{GROUP BY operator applies to the entire sales table in the view}. In the transformed query, the joins to products and customers \textbf{filter out a large portion of the rows} from the sales table, so the \textbf{GROUP BY operation is lower cost}. The join is more expensive because the sales table has not been reduced, but it is not much more expensive because the GROUP BY operation does not reduce the size of the row set very much in the original query. If any of the preceding characteristics were to change, merging the view might no longer be lower cost.


\noindent The following query of the \texttt{cust\_prod\_v} view uses a \texttt{DISTINCT} operator:
\begin{verbatim}
SELECT c.cust_id, c.cust_first_name, c.cust_last_name, c.cust_email
FROM customers c, products p,
(SELECT DISTINCT s.cust_id, s.prod_id FROM sales s) cust_prod_v
WHERE c.country_id = 52790 AND c.cust_id = cust_prod_v.cust_id
AND cust_prod_v.prod_id = p.prod_id
AND p.prod_name = 'T3 Faux Fur-Trimmed Sweater';
\end{verbatim}

\noindent After determining that view merging produces a lower-cost plan, the optimizer rewrites the query into this equivalent query:
\begin{verbatim}
SELECT nwv.cust_id, nwv.cust_first_name, nwv.cust_last_name, nwv.cust_email
FROM ( SELECT DISTINCT(c.rowid), p.rowid, s.prod_id, s.cust_id,
c.cust_first_name, c.cust_last_name, c.cust_email
FROM customers c, products p, sales s
WHERE c.country_id = 52790 AND c.cust_id = s.cust_id
AND s.prod_id = p.prod_id
AND p.prod_name = 'T3 Faux Fur-Trimmed Sweater' ) nwv;
\end{verbatim}


\subsubsection{Predicate Pusing}

In predicate pushing, the optimizer "pushes" the
relevant predicates from the containing query block
into the view query block.
For views that are not merged, this technique
improves the subplan of the unmerged view. The
database can use the pushed-in predicates to access
indexes or to use as filters.


\subsubsection{Join Factorization}


\noindent In the cost-based transformation known as \textbf{join factorization}, the optimizer can factorize common computations from branches of a \texttt{UNION ALL} query.

\vspace{0.5em}

\noindent \texttt{UNION ALL} queries are common in database applications, especially in data integration applications. Often, branches in a \texttt{UNION ALL} query refer to the same base tables. Without join factorization, the optimizer evaluates each branch of a \texttt{UNION ALL} query independently, which leads to repetitive processing, including data access and joins. Join factorization transformation can share common computations across the \texttt{UNION ALL} branches. Avoiding an extra scan of a large base table can lead to a huge performance improvement.

\vspace{0.5em}

\noindent Join factorization can factorize multiple tables and from more than two \texttt{UNION ALL} branches.


\begin{verbatim}
SELECT t1.c1, t2.c2
FROM t1, t2, t3
WHERE t1.c1 = t2.c1
AND t1.c1 > 1
AND t2.c2 = 2
AND t2.c2 = t3.c2
UNION ALL
SELECT t1.c1, t2.c2
FROM t1, t2, t4
WHERE t1.c1 = t2.c1
AND t1.c1 > 1
AND t2.c3 = t4.c3
\end{verbatim}

Without any transformation, the database must perform the scan and the filtering on table t1 twice, one time for each branch. 


\begin{verbatim}
SELECT t1.c1, VW_JF_1.item_2
FROM t1, (SELECT t2.c1 item_1, t2.c2 item_2
FROM t2, t3
WHERE t2.c2 = t3.c2
AND t2.c2 = 2
UNION ALL
SELECT t2.c1 item_1, t2.c2 item_2
FROM t2, t4
WHERE t2.c3 = t4.c3) VW_JF_1
WHERE t1.c1 = VW_JF_1.item_1 AND t1.c1 > 1
\end{verbatim}

In this case, because table t1 is factorized, the database performs the
table scan and the filtering on t1 only one time. If t1 is large, then this factorization avoids the huge performance cost of scanning and filtering
t1 twice.

\subsection{SQL Operators: Access Paths and Joins}


A row source is a set of rows returned by a step in the
execution plan. A SQL operator acts on a row source.
A unary operator acts on one input, as with access paths.
A binary operator acts on two outputs, as with joins.
An access path is a technique used by a query to retrieve
rows from a row source.

A row source is a set of rows returned by a step in an
execution plan. A row source can be a table, view, or
result of a join or grouping operation.
A unary operation such as an access path, which is a
technique used by a query to retrieve rows from a row
source, accepts a single row source as input. For example,
a full table scan is the retrieval of rows of a single row
source. In contrast, a join is binary and receives inputs
from exactly two row sources
The database uses different access paths for different
relational data structures. The following table summarizes
common access paths for the major data structures.


In general, index access paths are more efficient for
statements that retrieve a small subset of table rows,
whereas full table scans are more efficient when accessing
a large portion of a table.

\subsubsection{Table Access Paths}

A table is the basic unit of data organization in an Oracle
database. Relational tables are the most common table
type. Relational tables have with the following
organizational characteristics:

\begin{itemize}
  \item A heap-organized table does not store rows in any
particular order.
  \item An index-organized table orders rows according to the
primary key values.
  \item An external table is a read-only table whose metadata is
stored in the database but whose data is stored outside
the database.
\end{itemize}

\textbf{Heap-Organized Table Access}

By default, a table is organized as a heap, which means that
the database places rows where they fit best rather than in a
user-specified order.
As users add rows, the database places the rows in the first
available free space in the data segment. Rows are not
guaranteed to be retrieved in the order in which they were
inserted.
Every row in a heap-organized table has a rowid unique to this
table that corresponds to the physical address of a row piece.
A rowid is a 10-byte physical address of a row. Oracle Database
uses rowids internally for the construction of indexes.


A rowid is an internal representation of the storage location of
data. The rowid of a row specifies the data file and data block
containing the row and the location of the row in that block.
Locating a row by specifying its rowid is the fastest way to
retrieve a single row because it specifies the exact location of
the row in the database.
In most cases, the database accesses a table by rowid after a
scan of one or more indexes. However, table access by rowid
need not follow every index scan. If the index contains all
needed columns, then access by rowid might not occur.

A \textbf{sample table scan} retrieves a random sample of data
from a simple table or a complex SELECT statement, such
as a statement involving joins and views.
The database uses a sample table scan when a statement
FROM clause includes the SAMPLE keyword.

\textbf{B-Tree Index Access Paths}


An index is an optional structure, associated with a table or
table cluster, that can sometimes speed data access.
By creating an index on one or more columns of a table, you
gain the ability in some cases to retrieve a small set of
randomly distributed rows from the table. Indexes are one of
many means of reducing disk I/O.
B-trees, short for balanced trees, are the most common type
of database index.
A B-tree index is an ordered list of values divided into ranges.
By associating a key with a row or range of rows, B-trees
provide excellent retrieval performance for a wide range of
queries, including exact match and range searches.


\noindent D-4: Index Access Paths

\vspace{0.5em}

\noindent \textbf{Index Unique Scans}: An index unique scan returns at most 1 \texttt{rowid}.

\vspace{0.5em}

\noindent \textbf{Index Range Scans}: An index range scan is an ordered scan of values. The optimizer typically chooses a range scan for queries with \textbf{high selectivity}. By default, the database stores indexes in ascending order, and scans them in the same order. An \textbf{index range scan descending} is identical to an index range scan except that the database returns rows in descending order.

\vspace{0.5em}

\noindent \textbf{Index Full Scans}: An index full scan reads the entire index in order. An index full scan can eliminate a separate sorting operation because the data in the index is ordered by index key.

\textbf{Bitmap Index Access Paths}

Bitmap indexes combine the indexed data with a rowid
range.
In a conventional B-tree index, one index entry points to a
single row. In a bitmap index, the key is the combination
of the indexed data and the rowid range.
The database stores at least one bitmap for each index
key. Each value in the bitmap, which is a series of 1 and 0
values, points to a row within a rowid range. Thus, in a
bitmap index, one index entry points to a set of rows
rather than a single row.

\begin{itemize}
  \item Bitmap Index Single Value: This type of access path uses a
bitmap index to look up a single key value.
  \item Bitmap Index Range Scans: This type of access path uses a
bitmap index to look up a range of values.
  \item Bitmap Merge: This access path merges multiple bitmaps,
and returns a single bitmap as a result. A bitmap merge is
indicated by the BITMAP MERGE operation in an
execution plan.
\end{itemize}


\textbf{Table Cluster Access Paths}

A table cluster is a group of tables that share common
columns and store related data in the same blocks. When
tables are clustered, a single data block can contain rows
from multiple tables.
Cluster Scans: An index cluster is a table cluster that uses
an index to locate data
Hash Scans: A hash cluster is like an indexed cluster,
except the index key is replaced with a hash function. No
separate cluster index exists. In a hash cluster, the data is
the index. The database uses a hash scan to locate rows in
a hash cluster based on a hash value.


\textbf{Joins}

A join combines the output from exactly two row sources,
such as tables or views, and returns one row source. The
returned row source is the data set.
Whenever multiple tables exist in the FROM clause,
Oracle Database performs a join.
A join condition compares two row sources using an
expression. The join condition defines the relationship
between the tables. If the statement does not specify a
join condition, then the database performs a Cartesian
join, matching every row in one table with every row in
the other table.


\noindent The following query joins the \texttt{hr.employees} table with the \texttt{dept\_locs\_v} view, which returns the street address for each department. \texttt{dept\_locs\_v} is a join of the departments and locations tables.
\begin{verbatim}
SELECT e.first_name, e.last_name, dept_locs_v.street_address, dept_locs_v.postal_code
FROM employees e, (SELECT d.department_id, d.department_name, l.street_address, l.postal_code
   FROM departments d, locations l
   WHERE d.location_id = l.location_id ) dept_locs_v
WHERE dept_locs_v.department_id = e.department_id AND e.last_name = 'Smith';
\end{verbatim}

\noindent The database can execute the preceding query by joining departments and locations to generate the rows of the view, and then joining this result to employees. Because the query contains the view \texttt{dept\_locs\_v}, and this view contains two tables, the optimizer must use one of the following join orders:

\begin{itemize}
    \item employees, \texttt{dept\_locs\_v} (departments, locations)
    \item employees, \texttt{dept\_locs\_v} (locations, departments)
    \item \texttt{dept\_locs\_v} (departments, locations), employees
    \item \texttt{dept\_locs\_v} (locations, departments), employees
\end{itemize}

\noindent Join methods are also constrained. The index-based nested loops join is not feasible for join orders that begin with employees because no index exists on the column from this view. Without view merging, the optimizer generates the following execution plan:

\begin{itemize}
  \item \textbf{Nested Loops Joins}: Nested loops join an outer data set to an inner data
set. For each row in the outer data set that matches the single-table
predicates, the database retrieves all rows in the inner data set that satisfy the join predicate. If an index is available, then the database can use it to
access the inner data set by rowid.
  \item \textbf{Hash Joins}: The database uses a hash join to join larger data sets. The
optimizer uses the smaller of two data sets to \textbf{build a hash table on the
join key in memory}, using a deterministic hash function to specify the
location in the hash table in which to store each row. The database then
scans the larger data set, probing the hash table to find the rows that meet the join condition
  \item \textbf{Sort Merge Joins}: A sort merge join is a variation on a nested loops join. If
the two data sets in the join are not already sorted, then the database
\textbf{sorts} them. These are the SORT JOIN operations. For each row in the first
data set, the database probes the second data set for matching rows and
joins them, basing its start position on the match made in the previous iteration.
\end{itemize}


\subsection{Optimizer Controls}

\section{Software development}


There are two kinds of software products:

\begin{itemize}
  \item \textbf{Generic products}: These are stand-alone
systems that are produced by a development
organization and sold on the open market to
any customer who is able to buy them.
  \item \textbf{Customized (or bespoke) products}: These are
systems that are commissioned by a particular
customer. A software contractor develops the
software especially for that customer.
\end{itemize}



A \textbf{software process} is a sequence of activities that
leads to the production of a software product.

Four fundamental activities that are common to all
software processes:

\begin{itemize}
  \item \textbf{Software specification}, where customers and engineers
define the software that is to be produced and the
constraints on its operation.
  \item \textbf{Software development}, where the software is designed
and programmed.
  \item \textbf{Software validation}, where the software is checked to
ensure that it is what the customer requires.
  \item \textbf{Software evolution}, where the software is modified to
reflect changing customer and market requirements.
\end{itemize}


\subsection{Software process models}

\begin{itemize}
  \item \textbf{The waterfall model}: This takes the fundamental process
activities of specification, development, validation, and
evolution and represents them as separate process phases.

  \item \textbf{Incremental development}: This approach interleaves the
activities of specification, development, and validation. The
system is developed as a series of versions (increments),
with each version adding functionality to the previous
version.

  \item \textbf{Reuse-oriented software engineering}: This approach is
based on the existence of a significant number of reusable
components. The system development process focuses on
integrating these components into a system rather than
developing them from scratch.
\end{itemize}


\subsubsection{The waterfall model}

Plan-driven process: you must plan and
schedule all of the process activities before
starting work on them.

\begin{enumerate}
  \item \textit{Requirements analysis and definition}: The system's services, constraints, and goals are
established by consultation with system users. They are
then defined in detail and serve as a system specification.


  \item \textit{System and software design}: The systems design process allocates the requirements to
either hardware or software systems by establishing an
overall system architecture. Software design involves
identifying and describing the fundamental software
system abstractions and their relationships.

  \item \textit{Implementation and unit testing}: During this stage, the software design is realized as a set
of programs or program units. Unit testing involves
verifying that each unit meets its specification.

  \item \textit{Integration and system testing}: The individual program units or programs are
integrated and tested as a complete system to
ensure that the software requirements have been
met. After testing, the software system is delivered
to the customer.

  \item \textit{Operation and maintenance}: Normally (although not necessarily), this is the
longest life cycle phase. The system is installed and
put into practical use. Maintenance involves
correcting errors which were not discovered in
earlier stages of the life cycle, improving the
implementation of system units and enhancing the
system's services as new requirements are
discovered.
\end{enumerate}


In principle, the result of each phase is one or
more documents that are approved ('signed off').

The following phase should not start until the
previous phase has finished. In practice, these
stages overlap and feed information to each
other.


Documents produced in each phase may then
have to be modified to reflect the changes made.

Because of the costs of producing and
approving documents, iterations can be costly
and involve significant rework.

The probelms of the model may cause that
the system won't do what the user wants.

It may also lead to badly structured systems as
design problems are circumvented by
implementation tricks.

\subsubsection{Incremental development}

Incremental development is based on the idea
of developing an initial implementation,
exposing this to user comment and evolving it
through several versions until an adequate
system has been developed.

Specification, development, and validation
activities are interleaved rather than separate,
with rapid feedback across activities.

It is a fundamental part of agile approaches

It is better than a waterfall approach for most business,
e-commerce, and personal systems.

Incremental development reflects the way that we
solve problems: We rarely work out a complete
problem solution in advance but move toward a
solution in a series of steps, backtracking when we
realize that we have made a mistake.

By developing the software incrementally, it is cheaper
and easier to make changes in the software as it is
being developed.

Each increment or version of the system incorporates
some of the functionality that is needed by the
customer.

Generally, the early increments of the system include
the most important or most urgently required
functionality.


This means that the customer can evaluate the system
at a relatively early stage in the development to see if it
delivers what is required. If not, then only the current
increment has to be changed and, possibly, new
functionality defined for later increments.

Incremental development has three important benefits,
compared to the waterfall model:

\begin{itemize}
  \item The cost of accommodating changing customer
requirements is reduced. The amount of analysis and
documentation that has to be redone is much less than is
required with the waterfall model.
  \item It is easier to get customer feedback on the development
work that has been done.

  \item More rapid delivery and deployment of useful software to
the customer is possible, even if all of the functionality
has not been included. Customers are able to use and gain
value from the software earlier than is possible with a
waterfall process.
\end{itemize}


The incremental approach has two problems:

\begin{itemize}
  \item The process is not visible. Managers need regular
deliverables to measure progress. If systems are
developed quickly, it is not cost-effective to produce
documents that reflect every version of the system.

  \item System structure tends to degrade as new increments
are added. Unless time and money is spent on
refactoring to improve the software, regular change
tends to corrupt its structure. Incorporating further
software changes becomes increasingly difficult and
costly.
\end{itemize}

The problems of incremental development
become particularly acute for large, complex,
long-lifetime systems, where different teams
develop different parts of the system.

Large systems need a \textbf{stable framework or
architecture} and the responsibilities of the
different teams working on parts of the system
need to be clearly defined with respect to that
architecture. This has to be planned in advance
rather than developed incrementally.

\subsubsection{Reuse-oriented software engineering}

People working on the project know of designs or code
that are similar to what is required. They look for
these, modify them as needed, and incorporate them
into their system.


Reuse-oriented approaches rely on a large base of
reusable software components and an integrating
framework for the composition of these components.
Sometimes, these components are systems in their
own right (COTS or commercial off-the-shelf systems)
that may provide specific functionality such as word
processing or a spreadsheet.


Stages:


\begin{enumerate}
  \item \textit{Requirements specification}
  \item \textit{Component analysis}: Given the requirements specification, a search is made for components to implement that
specification. Usually, there is no exact match and the components that may be used only
provide some of the functionality required.
  \item \textit{Requirements modification}: During this stage, the requirements are analyzed using information about the components
that have been discovered. They are then modified to reflect the available components.
Where modifications are impossible, the component analysis activity may be re-entered to
search for alternative solutions.

  \item \textit{System design with reuse}: During this phase, the framework of the system is designed or an existing framework is
reused. The designers take into account the components that are reused and organize the
framework to cater for this. Some new software may have to be designed if reusable
components are not available.

  \item \textit{Development and integration}: Software that cannot be externally procured is developed, and the components and COTS
systems are integrated to create the new system. System integration, in this model, may be
part of the development process rather than a separate activity.

  \item \textit{Validation}
\end{enumerate}


There are three types of software component
that may be used in a reuse-oriented process:

\begin{itemize}
  \item Web services that are developed according to
service standards and which are available for
remote invocation.

  \item Collections of objects that are developed as a
package to be integrated with a component
framework such as .NET or J2EE.

  \item Stand-alone software systems that are
configured for use in a particular
environment.
\end{itemize}


\subsection{Process activities}


The activities in the design process vary, depending on the
type of system being developed. The four activities that may
be part of the design process for information systems:


\begin{enumerate}
  \item \textit{Architectural design}, where you identify the overall
structure of the system, the principal components
(sometimes called sub-systems or modules), their
relationships, and how they are distributed.

  \item \textit{Interface design}, where you define the interfaces between
system components. This interface specification must be
unambiguous. With a precise interface, a component can
be used without other components having to know how it
is implemented. Once interface specifications are agreed,
the components can be designed and developed
concurrently.

  \item \textit{Component design}, where you take each system component and
design how it will operate. This may be a simple statement of the
expected functionality to be implemented, with the specific
design left to the programmer. Alternatively, it may be a list of
changes to be made to a reusable component or a detailed design
model. The design model may be used to automatically generate
an implementation.


  \item \textit{Database design}, where you design the system data structures
and how these are to be represented in a database. Again, the
work here depends on whether an existing database is to be
reused or a new database is to be created.
\end{enumerate}


\section{Architectural Design}

You can design software architectures at two levels of
abstraction:

\begin{itemize}
  \item \textbf{Architecture in the small} is concerned with the
architecture of individual programs. At this level, we
are concerned with the way that an individual program
is decomposed into components.

  \item \textbf{Architecture in the large} is concerned with the
architecture of complex enterprise systems that
include other systems, programs, and program
components. These enterprise systems are distributed
over different computers, which may be owned and
managed by different companies.
\end{itemize}

System architectures are often modeled using simple
block diagrams.

Each box in the diagram represents a component.

Boxes within boxes indicate that the component has
been decomposed to sub-components.


Arrows mean that data and or control signals are
passed from component to component in the direction
of the arrows.

Block diagrams present a high-level picture of the
system structure, which people from different disciplines, who are involved in the
system development process, can readily understand.


There are two ways in which an architectural model of a program is
used:


\begin{itemize}
  \item As a way of facilitating discussion about the system design
  \item As a way of documenting an architecture that has been designed
\end{itemize}


Because of the close relationship between non-
functional requirements and software architecture,
the particular architectural style and structure that
you choose for a system should depend on the non-
functional system requirements:


\begin{itemize}
  \item Performance
  \item Security
  \item Safety (event of faliure)
  \item Availability
  \item Maintainability
\end{itemize}


It is suggested that there should be four fundamental architectural views:

\begin{itemize}
  \item \textbf{A logical view}: which shows the key abstractions in the system as objects or object classes. It
should be possible to relate the system requirements to entities in this logical
view.
  \item \textbf{A process view}: which shows how, at run-time, the system is composed of interacting processes.
This view is useful for making judgments about nonfunctional system
characteristics such as performance and availability.
  \item \textbf{A development view}: which shows how the software is decomposed for development, that is, it shows
the breakdown of the software into components that are implemented by a single
developer or development team. This view is useful for software managers and
programmers.

  \item \textbf{A physical view}: which shows the system hardware and how software components are distributed
across the processors in the system. This view is useful for systems engineers
planning a system deployment.
\end{itemize}


\subsection{Architectural Patterns}

The idea of patterns as a way of presenting, sharing, and
reusing knowledge about software systems is now widely
used.

You can think of an architectural pattern as a stylized,
abstract description of good practice, which has been
tried and tested in different systems and environments.

So, an architectural pattern should describe a system
organization that has been successful in previous systems.


It should include information of when it is and is not
appropriate to use that pattern, and the pattern's
strengths and weaknesses.



\subsubsection{Layered Architecture}


The system functionality is organized into separate layers, and each layer only
relies on the facilities and services offered by the layer immediately beneath
it.

This layered approach supports the incremental development of systems.

As a layer is developed, some of the services provided by that layer may be
made available to users.


The architecture is changeable and portable. So long as its interface is
unchanged, a layer can be replaced by another, equivalent layer.

When layer interfaces change or new facilities are added to a layer, only the
adjacent layer is affected.


\begin{figure}[H]  % or [!htbp] if you’re fine with floating
  \centering
  \includegraphics[width=1\textwidth]{layered_architecture.png} % adjust width/height as needed
\end{figure}

\begin{itemize}
  \item \textbf{Description}: Organizes the system into layers with related functionality
associated with each layer. A layer provides services to the layer above it so
the lowest-level layers represent core services that are likely to be used
throughout the system.

  \item \textbf{Example}: A layered model of a system for sharing copyright documents held
in different libraries.

  \item \textbf{When used}: Used when building new facilities on top of existing systems;
when the development is spread across several teams with each team
responsibility for a layer of functionality; when there is a requirement for
multi-level security.

  \item \textbf{Advantages}: Allows replacement of entire layers so long as the interface is
maintained. Redundant facilities (e.g., authentication) can be provided in
each layer to increase the dependability of the system.


  \item \textbf{Disadvantages}: In practice, providing a clean separation between layers is
often difficult and a high-level layer may have to interact directly with lower-
level layers rather than through the layer immediately below it. Performance
can be a problem because of multiple levels of interpretation of a service
request as it is processed at each layer.
\end{itemize}



\subsubsection{Repository Architecture}


It describes how a set of interacting components
can share data.


The majority of systems that use large amounts
of data are organized around a shared database
or repository. This model is therefore suited to
applications in which data is generated by one
component and used by another.


\begin{figure}[H]  % or [!htbp] if you’re fine with floating
  \centering
  \includegraphics[width=1\textwidth]{repository_architecture.png} % adjust width/height as needed
\end{figure}

\begin{itemize}
  \item \textbf{Description}: All data in a system is managed in a central repository that is
accessible to all system components. Components do not interact directly,
only through the repository.

  \item \textbf{Example}: command and control systems, management information systems,
CAD systems, and interactive development environments for software.

  \item \textbf{When used}: You should use this pattern when you have a system in which
large volumes of information are generated that has to be stored for a long
time. You may also use it in data-driven systems where the inclusion of data in
the repository triggers an action or tool.


  \item \textbf{Advantages}: Components can be independent—they do not need to know of
the existence of other components. Changes made by one component can be
propagated to all components. All data can be managed consistently (e.g.,
backups done at the same time) as it is all in one place.


  \item \textbf{Disadvantages}: The repository is a single point of failure so problems in the
repository affect the whole system. May be inefficiencies in organizing all
communication through the repository. Distributing the repository across
several computers may be difficult.

\end{itemize}


\subsubsection{Client-server Architecture}


A system that follows the client-server pattern is
organized as a set of services and associated servers, and
clients that access and use the services.

The major components of this model are:

\begin{itemize}
  \item A set of servers that offer services to other components.
  \item A set of clients that call on the services offered by servers.
There will normally be several instances of a client program
executing concurrently on different computers.
  \item A network that allows the clients to access these services.
Most client-server systems are implemented as distributed
systems, connected using Internet protocols.
\end{itemize}


\begin{itemize}
  \item \textbf{Description}: The functionality of the system is organized into services, with
each service delivered from a separate server. Clients are users of these
services and access servers to make use of them.

  \item \textbf{Example}: A film and video/DVD library organized as a client-server system.
  
  \item \textbf{When used}: Used when data in a shared database has to be accessed from a
range of locations. Because servers can be replicated, may also be used when
the load on a system is variable.

  \item \textbf{Advantages}: The principal advantage of this model is that servers can be
distributed across a network. General functionality (e.g., a printing service)
can be available to all clients and does not need to be implemented by all
services.

  \item \textbf{Disadvantages}: Each service is a single point of failure so susceptible to denial
of service attacks or server failure. Performance may be unpredictable
because it depends on the network as well as the system. May be
management problems if servers are owned by different organizations.

\end{itemize}


\subsubsection{Pipe and filter architecture (Event-Driven)}


This is a model of the run-time organization of a system where
functional transformations process their inputs and produce
outputs.

Data flows from one to another and is transformed as it
moves through the sequence.

Each processing step is implemented as a transform.

Input data flows through these transforms until converted to
output.


The transformations may execute sequentially or in parallel.

The data can be processed by each transform item by item or
in a single batch.


\begin{itemize}
  \item \textbf{Description}: The processing of the data in a system is organized so that each
processing component (filter) is discrete and carries out one type of data
transformation. The data flows (as in a pipe) from one component to another
for processing.

  \item \textbf{Example}: A billing system used for processing invoices.
  
  \item \textbf{When used}: Commonly used in data processing applications (both batch- and
transaction-based) where inputs are processed in separate stages to generate
related outputs.

  \item \textbf{Advantages}: Easy to understand and supports transformation reuse.
Workflow style matches the structure of many business processes. Evolution
by adding transformations is straightforward. Can be implemented as either a
sequential or concurrent system.


  \item \textbf{Disadvantages}: The format for data transfer has to be agreed upon between
communicating transformations. Each transformation must parse its input
and unparse its output to the agreed form. This increases system overhead
and may mean that it is impossible to reuse functional transformations that
use incompatible data structures.
\end{itemize}


\subsubsection{MVC Pattern}

This pattern is the basis of interaction
management in many web-based systems.

\begin{figure}[H]  % or [!htbp] if you’re fine with floating
  \centering
  \includegraphics[width=1\textwidth]{mvc_pattern.png} % adjust width/height as needed
\end{figure}

\begin{itemize}
  \item \textbf{Description}: Separates presentation and interaction from the system data.
The system is structured into

  \item three logical components that interact with each other. The Model
component manages the system data and associated operations on that data.
The View component defines and manages how the data is presented to the
user. The Controller component manages user interaction (e.g., key presses,
mouse clicks, etc.) and passes these interactions to the View and the Model.

  \item \textbf{Example}: The architecture of a web-based application system organized using
the MVC pattern.

  \item \textbf{When used}: Used when there are multiple ways to view and interact with
data. Also used when the future requirements for interaction and
presentation of data are unknown.

  \item \textbf{Advantages}: Allows the data to change independently of its representation
and vice versa. Supports presentation of the same data in different ways with
changes made in one representation shown in all of them.

  \item \textbf{Disadvantages}: Can involve additional code and code complexity when the
data model and interactions are simple.
\end{itemize}


\end{document}
